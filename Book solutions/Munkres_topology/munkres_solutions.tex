% --------------------------------------------------------------
% This is all preamble stuff that you don't have to worry about.
% Head down to where it says "Start here"
% --------------------------------------------------------------

\documentclass[12pt]{article}

\usepackage[margin=1in]{geometry}
\usepackage{amsmath,amsthm,amssymb,scrextend}
\usepackage{fancyhdr}
\pagestyle{fancy}
\usepackage{xcolor}
\usepackage{enumitem}

\newcommand{\N}{\mathbb{N}}
\newcommand{\Z}{\mathbb{Z}}
\newcommand{\I}{\mathbb{I}}
\newcommand{\R}{\mathbb{R}}
\newcommand{\Q}{\mathbb{Q}}
\renewcommand{\qed}{\hfill$\blacksquare$}
\let\newproof\proof
\renewenvironment{proof}{\begin{addmargin}[1em]{0em}\begin{newproof}}{\end{newproof}\end{addmargin}\qed}
% \newcommand{\expl}[1]{\text{\hfill[#1]}$}

\newenvironment{theorem}[2][Theorem]{\begin{trivlist}
\item[\hskip \labelsep {\bfseries #1}\hskip \labelsep {\bfseries #2}]}{\end{trivlist}}
\newenvironment{lemma}[2][Lemma]{\begin{trivlist}
\item[\hskip \labelsep {\bfseries #1}\hskip \labelsep {\bfseries #2.}]}{\end{trivlist}}
\newenvironment{problem}[2][Exercise]{\begin{trivlist}
\item[\hskip \labelsep {\bfseries #1}\hskip \labelsep {\bfseries #2.}]}{\end{trivlist}}
\newenvironment{exercise}[2][Exercise]{\begin{trivlist}
\item[\hskip \labelsep {\bfseries #1}\hskip \labelsep {\bfseries #2.}]}{\end{trivlist}}
\newenvironment{reflection}[2][Reflection]{\begin{trivlist}
\item[\hskip \labelsep {\bfseries #1}\hskip \labelsep {\bfseries #2.}]}{\end{trivlist}}
\newenvironment{proposition}[2][Proposition]{\begin{trivlist}
\item[\hskip \labelsep {\bfseries #1}\hskip \labelsep {\bfseries #2.}]}{\end{trivlist}}
\newenvironment{corollary}[2][Corollary]{\begin{trivlist}
\item[\hskip \labelsep {\bfseries #1}\hskip \labelsep {\bfseries #2.}]}{\end{trivlist}}



\begin{document}

% --------------------------------------------------------------
%                         Start here
% --------------------------------------------------------------

\lhead{Munkres ``Topology'' Solutions}
\chead{Stan Tuznik}
\rhead{\today}

% \maketitle


\section*{Chapter 1. Set Theory and Logic}

\subsection*{1.1 Fundamental Concepts}


\begin{problem}{1.1.1}
Check the distributive laws for $\cup$ and $\cap$ and DeMorgan's laws.
\end{problem}

\begin{proof}
The laws in question are
\begin{equation*}
	\begin{split}
		A \cap \left(B\cup C\right) & = \left(A\cap B\right)\cup \left(A\cap C\right) \\
		A \cup \left(B\cap C\right) & = \left(A \cup B\right) \cap \left(A\cup C\right) \\
		A \setminus \left(B\cup C\right) & = \left(A\setminus B\right)\cap \left(A\setminus C\right)\\
		A \setminus \left(B\cap C\right) & = \left(A\setminus B\right)\cup \left(A\setminus C\right) \\
	\end{split}
\end{equation*}
We can show that any one of these equations is true by considering a point in the set on the left-hand side of the equals sign and showing that it is also in the set on the right-hand side of the equals sign. Then we proceed in the other direction. 

For the first equation, let $x\in A\cap \left(B\cup C\right)$. Then $x\in A$ and $x\in B\cup C$. If $x\in B$, then $x\in A\cap B$. Otherwise, if $x\in C$, then $x \in A\cap C$. In either case, then, we have $x \in \left(A\cap B\right)\cup \left(A\cap C\right)$. Conversely, let $x\in \left(A\cap B\right)\cup \left(A\cap C\right)$. Then $x \in A\cap B$ or $x \in A\cap C$. In either case, $x\in A$. Then we either have $x \in B$ or $x\in C$. Thus, $ x \in A\cap \left(B\cup C\right)$. By double inclusion, we have shown that the first law holds.

The proof of the second law is very similar, and so we omit it.

For the first of DeMorgan's laws, we first let $x \in A\setminus \left(B\cup C\right)$. Then $x\in A$ but $x\notin B\cup C$. Then $x\notin B$ and $x\notin C$. Then $x\in A\setminus B$ and $x\in A\setminus C$, and so the first direction holds. The converse is also easy to show.

For the second of DeMorgan's laws, let $x\in A\setminus \left(B\cap C\right)$. Then $x\in A$ and $x\notin B\cap C$. Then either $x\notin B$ or $x\notin C$. So either $x\in A\setminus B$ or $x\in A\setminus C$, and so $x \in \left(A\setminus B\right)\cup\left(A\setminus C\right)$. The converse is easily shown to hold as well.
\end{proof}

\begin{problem}{1.1.3}
\begin{enumerate}[label=(\alph*)] 
	\item Write the contrapositive and converse of the following statement: ``If $x<0$, then $x^2 -x >0$,'' and determine which (if any) of the three statements are true.
	\item Do the same for the statement ``If $x>0$, then $x^2-x>0$.''
\end{enumerate}
\end{problem}
\begin{proof}
\begin{enumerate}[label=(\alph*)]
	\item The contrapositive is ``If $x^2-x\leq 0$, then $x \leq 0$.'' The converse is ``If $x^2-x>0$, then $x>0$.'' The contrapositive is certainly false, as $x=1/2$ gives $x^2-x = 1/4-1/2 = -1/4$ but $x > 0$. Similarly, the converse is false, since $x = -2$ satisfies $x^2-x>0$ but $x<0$. The original statement itself is false, since $x=-1/2$ is a counterexample.
	\item The contrapositive is ``If $x^2-x \leq 0$, then $x \leq 0$.'' The converse is ``If $x^2-x >0$, then $x>0$. The original statement is false, since $x=1/2$ is a counterexample. The contrapositive is false, since $x=1/2$ is a counterexample. Lastly, $x=-3$ is a counterexample for the converse.
\end{enumerate}
\end{proof}

\begin{problem}{1.1.4}
Let $A$ and $B$ be sets of real numbers. Write the negation of each of the following statements.
\begin{enumerate}[label=(\alph*)]
	\item For every $a\in A$, it is true that $a^2 \in B$.
	\item For at least one $a\in A$, it is true that $a^2 \in B$.
	\item For every $a\in A$, it is true that $a^2 \notin B$.
	\item For at least one $a\notin A$, it is true that $a^2 \in B$.
\end{enumerate}
\end{problem}
\begin{proof}
\begin{enumerate}[label=(\alph*)]
	\item There is some $a\in A$ with $a^2 \notin B$.
	\item For all $a\in A$, $a^2 \notin B$.
	\item There is some $a\in A$ with $a^2 \in B$.
	\item For all $a\notin A$, we have $a^2 \notin B$.
\end{enumerate}
\end{proof}



\begin{problem}{1.1.5}
Let $\mathcal{A}$ be a nonempty collection of sets. Determine the truth of each of the following statements and of their converses:
\begin{enumerate}[label=(\alph*)]
	\item $x\in \cup_{A\in \mathcal{A}} A \implies x\in A$ for at least one $A\in \mathcal{A}$.
	\item $x \in \cup_{A\in \mathcal{A}} A \implies x\in A$ for every $A \in \mathcal{A}$.
	\item $x \in \cap_{A\in \mathcal{A}} A \implies x\in A$ for at least one $A \in \mathcal{A}$.
	\item $x \in \cap_{A\in \mathcal{A}} A \implies x \in A$ for every $A \in \mathcal{A}$.
\end{enumerate}
\end{problem}
\begin{proof}
\begin{enumerate}[label=(\alph*)]
	\item This implication is true by the very definition of the union of sets. The converse is also true, also by the same definition.
	\item This implication is not true, since it would be the definition of intersection. The converse is certainly true.
	\item The implication is true, but the converse is not, since it is only in the converse if it is in every $A\in \mathcal{A}$.
	\item The statement is true and is the definition of intersection. The converse is also true and is the definition of intersection.
\end{enumerate}
\end{proof}


\begin{problem}{1.1.6}
Write the contrapositive of each of the statements of Exercise 5.
\end{problem}
\begin{proof}
\begin{enumerate}[label=(\alph*)]
	\item For every
\end{enumerate}
\end{proof}















\subsection*{1.2 Functions}

\begin{problem}{1.2.1}
Let $f:A\rightarrow B$. Let $A_0 \subset A$ and $B_0 \subset B$.
\begin{enumerate}[label=(\alph*)]
	\item Show that $A_0 \subset f^{-1}\left(f\left(A_0\right)\right)$ and that equality holds is $f$ is injective.
	\item Show that $f\left(f^{-1}\left(B_0\right)\right) \subset B_0$ and that equality holds if $f$ is surjective.
\end{enumerate}
\end{problem}
\begin{proof}
\begin{enumerate}[label=(\alph*)]
	\item Let $x\in A_0 \subset A$. Then $f\left(x\right) \in f\left(A_0\right)$ and so $x \in f^{-1}\left(f\left(A_0\right)\right)$. Hence, $A_0 \subset f^{-1}\left(f\left(A_0\right)\right)$. If $f$ injective, then let $x \in f^{-1}\left(f\left(A_0\right)\right)$. Since $x \in f^{-1}\left(f\left(A_0\right)\right)$, we know $f\left(x\right) \in f\left(A_0\right)$. But then $f\left(x\right) = f\left(x_0\right)$ for some $x_0 \in A_0$, by definition of $f\left(A_0\right)$. But then by injectivity we have $x = x_0 \in A_0$. Thus, we have $f^{-1}\left(f\left(A_0\right)\right) \subset A_0$ if $f$ injective, and so by double inclusion we have that equality holds.
	\item Let $y \in f\left(f^{-1}\left(B_0\right)\right)$. Then there is some $x\in f^{-1}\left(B_0\right)$ such that $f\left(x\right)=y$. But then since $x\in f^{-1}\left(B_0\right)$, we have $f\left(x\right) \in B_0$ and since $f\left(x\right)=y$, we have $y \in B_0$. Hence, $f\left(f^{-1}\left(B_0\right)\right) \subset B_0$. If $f$ surjective, let $y \in B_0$. Then there is some $x \in f^{-1}\left(B_0\right)$ with $f\left(x\right)=y$. But then $y \in f\left(f^{-1}\left(B_0\right)\right)$, and so $B_0 \subset f\left(f^{-1}\left(B_0\right)\right)$ when $f$ surjective. Equality holds in this case by double inclusion.
\end{enumerate}
\end{proof}


\begin{problem}{1.2.2}
Let $f:A\rightarrow B$ and let $A_i \subset A$ and $B_i \subset B$ for $i=0$ and $i=1$. Show that $f^{-1}$ preserves {\color{red}(i.e., distributes over)} inclusions, unions, intersections, and differences of sets:
\begin{enumerate}[label=(\alph*)]
	\item $B_0 \subset B_1 \implies f^{-1}\left( B_0 \right) \subset f^{-1} \left(B_1\right)$.
	\item $f^{-1}\left(B_0 \cup B_1\right) = f^{-1}\left(B_0\right)\cup f^{-1}\left(B_1\right)$.
	\item $f^{-1}\left(B_0\cap B_1\right) = f^{-1}\left(B_0\right) \cap f^{-1}\left(B_1\right)$.
	\item $f^{-1}\left(B_0-B_1\right) = f^{-1}\left(B_0\right) - f^{-1}\left(B_1\right)$.
\end{enumerate}
Show that $f$ preserves inclusions and unions only:
\begin{enumerate}[label=(\alph*)]
	\setcounter{enumi}{4}
	\item $A_0 \subset A_1 \implies f\left(A_0\right) \subset f\left(A_1\right)$.
	\item $f\left(A_0 \cup A_1\right) = f\left(A_0\right) \cup f\left(A_1\right)$.
	\item $f\left(A_0 \cap A_1\right) \subset f\left(A_0\right)\cap f\left(A_1\right)$; show that equality holds if $f$ is injective.
	\item $f\left(A_0 - A_1\right) \supset f\left(A_0\right)-f\left(A_1\right)$; show that equality holds if $f$ is injective.
\end{enumerate}
\end{problem}

\begin{proof}
Let $f$, $A_i$, and $B_i$ be given as in the problem hypotheses.
\begin{enumerate}[label=(\alph*)]
	\item Let $B_0 \subset B_1$. Then let $x \in f^{-1}\left(B_0\right)$. That is, $f\left(x\right) \in B_0 \subset B_1$, so $f\left(x\right) \in B_1$. Hence, $x \in f^{-1}\left(B_1\right)$. Thus, $f^{-1}\left(B_0\right) \subset f^{-1}\left(B_1\right)$.
	\item Let $x\in f^{-1}\left(B_0 \cup B_1\right)$. That is, $f\left(x\right) \in B_0 \cup B_1$. Then $f\left(x\right) \in B_0$ or $f\left(x\right) \in B_1$. In other words, $x\in f^{-1}\left(B_0\right)$ or $x \in f^{-1}\left(B_1\right)$, respectively. Hence, $x \in f^{-1}\left(B_0\right)\cup f^{-1}\left(B_1\right)$. Conversely, if $x \in  f^{-1}\left(B_0\right)\cup f^{-1}\left(B_1\right)$, then $x \in f^{-1}\left(B_0\right)$ or $x\in f^{-1}\left(B_1\right)$. In other words, $f\left(x\right) \in B_0$ or $f\left(x\right) \in B_1$, respectively. Then $f\left(x\right) \in B_0 \cup B_1$, and so $x \in f^{-1}\left(B_0 \cup B_1\right)$. By double inclusion, the equality holds.
	\item The proof of this statement is very similar to that of part (b).
	\item Let $x \in f^{-1}\left(B_0 - B_1 \right)$. That is, $f\left(x\right) \in B_0 - B_1$, so $f\left(x\right) \in B_0$ but $f\left(x\right) \notin B_1$. So $x \in f^{-1}\left(B_0\right)$ but $x \notin f^{-1}\left(B_1\right)$, so that $x \in f^{-1}\left(B_0\right) - f^{-1}\left(B_1\right)$. The proof of the converse is similar and straightforward.
	\item Let $A_0 \subset A_1$ and let $y \in f\left(A_0\right)$. Then there is an $x \in A_0$ with $f\left(x\right) = y$. But then $x \in A_1$, so $y \in f\left(A_1\right)$. Thus, $f\left(A_0\right) \subset f\left(A_1\right)$.
	\item Let $y \in f\left(A_0 \cup A_1\right)$. Then there is an $x \in A_0 \cup A_1$ with $f\left(x\right) = y$. If $ x \in A_0$, then $y \in f\left(A_0\right)$. Otherwise, if $x \in A_1$, then $y\in f\left(A_1\right)$. In any case, then, $y \in f\left(A_0\right)\cup f\left(A_1\right)$. The converse is just as straightforward.
	\item The proof of the set inclusion is just as straightforward as in (f). If $f$ injective, then let $y \in f\left(A_0\right)\cap f\left(A_1\right)$. Then $y \in f\left(A_0\right)$ and $y \in f\left(A_1\right)$. Then there are $x_0 \in A_0$ and $x_1 \in A_1$ such that $f\left(x_0\right) = y$ and $f\left(x_1\right)=y$. By transitivity, then, $f\left(x_0\right) = f\left(x_1\right)$, and by injectivity we have $x_0 = x_1$. Then $x_0 \in A_0 \cap A_1$, and so $y \in f^{-1}\left(A_0 \cap A_1\right)$. Hence, by double inclusion, inequality holds if $f$ injective.
	\item Let $y \in f\left(A_0\right) - f\left(A_1\right)$. Then $y \in f\left(A_0\right)$ and $y \notin f\left(A_1\right)$. Thus there is some $x \in A_0$ such that $f\left(x\right)=y$, but $f\left(z\right)\neq y$ for all $z \in A_1$. Thus, we must have $x \notin A_1$. Then $x \in A_0 - A_1$, and so $y \in f\left(A_0 -A_1\right)$. Thus, $f\left(A_0\right)-f\left(A_1\right) \subset f\left(A_0 - A_1\right)$. Assume $f$ injective, and let $y\in f\left(A_0 -A_1\right)$. Then there is some $x \in A_0 - A_1$ with $f\left(x\right)=y$. Thus $x \in A_0$ but $x\notin A_1$. Hence $y \in f\left(A_0\right)$. We may have $y \in f\left(A_1\right)$ even though $x\notin A_1$; for instance, assume that there is a $z\in A_1$ with $f\left(z\right)=y$. But then we have $f\left(z\right) = y = f\left(x\right)$, so we have $z = x$ by injectivity of $f$. But then $z=x \in A_0$ and $z=x \notin A_1$. $\bot$  Thus, no such $z$ can exist, and so $y \notin f\left(A_1\right)$. Hence, $y \in f\left(A_0\right)-f\left(A_1\right)$, and so double inclusion holds if $f$ injective.
\end{enumerate}
\end{proof}


\begin{problem}{1.2.4}
Let $f:A\rightarrow B$ and $g:B\rightarrow C$.
\begin{enumerate}[label=(\alph*)]
	\item If $C_0 \subset C$, show that $\left(g\circ f\right)^{-1}\left(C_0\right) = f^{-1}\left(g^{-1}\left(C_0\right)\right)$.
	\item If $f$ and $g$ are injective, show that $g \circ f$ is injective.
	\item If $g \circ f$ is injective, what can you say about injectivity of $f$ and $g$?
	\item If $f$ and $g$ are surjective, show that $g\circ f$ is surjective.
	\item If $g\circ f$ is surjective, what can you say about the surjectivity of $f$ and $g$?
	\item Summarize your answers to (b)-(e) in the form of a theorem.
\end{enumerate}
\end{problem}
\begin{proof}
\begin{enumerate}[label=(\alph*)]
	\item Let $C_0 \subset C$ and let $x \in \left(g \circ f\right)^{-1} \left(C_0\right)$. That is, $\left(g \circ f\right)\left(x\right) \in C_0$. By definition of composition of functions, $\left(g \circ f\right)\left(x\right) = g\left(f\left(x\right)\right) \in C_0$. Letting $y = f\left(x\right) \in B$, we have $g\left(y\right) \in C_0$, and so $y = f\left(x\right) \in g^{-1}\left(C_0\right)$. But then $x \in f^{-1}\left( g^{-1}\left(C_0\right)\right)$, hence $\left(g\circ f\right)^{-1}\left(C_0\right) \subset f^{-1}\left(g^{-1}\left(C_0\right)\right)$.
	
	Conversely, let $x\in f^{-1}\left(g^{-1}\left(C_0\right)\right)$. Then $f\left(x\right) \in g^{-1}\left(C_0\right)$, and thus $g\left(f\left(x\right)\right) \in C_0$. But $g\left(f\left(x\right)\right) = \left(g\circ f\right)\left(x\right) \in C_0$, so the other direction holds. By double inclusion, the equality is proven.
	
	\item Let $f$ and $g$ be injective and assume $g\circ f\left(x\right) = g \circ f \left(z\right)$ for some $x,z \in A$. Then by definition of function composition we have $g \left(f \left(x\right)\right) = g\left(f\left(z\right)\right)$ and by injectivity of $g$ we have $f\left(x\right) = f\left(z\right)$. Next, by injectivity of $f$ we have $x = z$, and so $g\circ f $ is injective.
	
	\item Let $g\circ f$ be injective. 
	
	\item Let $f$ and $g$ be surjective and let $y \in C$. Then since $g$ is surjective, there is a $z \in B$ with $g\left(z\right) = y$. Then since $z \in B$ and $f$ surjective, there is some $x \in A$ with $f\left(x\right) = z$. But then $y = g\left(z\right) = g\left(f\left(x\right)\right) = g \circ f \left(x\right)$ and so $g\circ f$ is surjective.
	
	\item Let $g\circ f$ be surjective.
	
	\item \begin{theorem}{}
	Hello.
	\end{theorem}
\end{enumerate}
\end{proof}



\begin{problem}{1.2.5}
In general, let us denote the \textbf{identity function} for a set $C$ by $i_C$. That is, define $i_C: C \rightarrow C$ to be the function given by the rule $i_C\left(x\right) = x$ for all $x \in C$. Given $f:A\rightarrow B$, we say that a function $g: B\rightarrow A$ is a \textbf{left inverse} for $f$ if $g\circ f = i_A$; and we say that $h:B\rightarrow A$ is a \textbf{right inverse} for $f$ if $f\circ h=i_B$.
\begin{enumerate}[label=(\alph*)]
	\item Show that if $f$ has a left inverse, $f$ is injective; and if $f$ has a right inverse, $f$ is surjective.
	\item Give an example of a function that has a left inverse but no right inverse.
	\item Give an example of a function that has a right inverse but no left inverse.
	\item Can a function have more than one left inverse? More than one right inverse?
	\item Show that if $f$ has both a left inverse $g$ and a right inverse $h$, then $f$ is bijective and $g=h=f^{-1}$.
\end{enumerate}
\end{problem}
\begin{proof}
\begin{enumerate}[label=(\alph*)]
	\item Let $f$ have a left inverse $g$. Then assume $f\left(x\right) = f\left(y\right)$ for some $x,y \in A$. Then by definition of inverse, we have $ g\circ f\left(x\right) = i_A\left(x\right) = x$. Further, by our hypothesis we have $g\circ f\left(x\right) = g\circ f\left(z\right) = i_A\left(z\right) = z$, so that $x = z$. Thus, $f$ is injective.
	
	Next, assume that $f$ has a right inverse $h$. Let $y \in B$. Then $h:B\rightarrow A$, so $h\left(y\right) \in A$. Now, note that since $f:A\rightarrow B$, we have $f\left(h\left(y\right)\right) \in B$. But by $h$ being a right inverse for $f$, we have $f\left(h\left(y\right)\right) = f\circ h \left(y\right) = i_B\left(y\right) = y$. Thus, $h\left(y\right)\in A$ is the element of $A$ which $f$ maps to $y \in B$; hence, $f$ is surjective.
	
	\item h
	
	\item
	
	\item Assume $f$ has two left inverses: $g_0$ and $g_1$. Then for $y \in f\left(A\right)$, the domain of any inverse of $f$, we have some $x\in A$ with $f\left(x\right) =y$. Then $g_0\left(y\right) = g_0\left(f\left(x\right)\right) = g_0 \circ f \left(x\right) = i_A\left(x\right)=x$ and $g_1\left(y\right) = g_1\left(f\left(x\right)\right) = g_1 \circ f\left(x\right) = i_A\left(x\right)=x$ so that $g_0\left(y\right) = g_1\left(y\right)$ for all $y \in f\left(A\right)$. Thus, $g_0 = g_1$ since it takes the same values on their domain. 
	Similar logic applies for right inverses, showing that both left and right inverses are unique.
	
	\item If $f$ has a left inverse $g$ and a right inverse $h$, then $f$ is bijective by parts (a) and (b) of this problem. We wish to show that $g = h$ by showing that the two functions take the same values on their entire domains; the domain of both $g$ and $h$ is the set $B$, since $f$ is a bijection from $A$ to $B$. Hence, let $y \in B$. Then by surjectivity of $f$ there is an $x \in A$ with $f\left(x\right) = y$. But then $g\left(y\right) = g\left(f\left(x\right)\right) = g\circ f \left(x\right) = i_A\left(x\right) = x$ by definition of $g$ as left inverse of $f$. But then $y = f\left(x\right) = f\left(g\left(y\right)\right)$. Next, consider $f\left(h\left(y\right)\right) = f\circ h \left(y\right) = i_B\left(y\right) = y$.  Thus, we have $f\left(h\left(y\right)\right) = y = f\left(g\left(y\right)\right)$, and so by injectivity of $f$ we have $h\left(y\right) = g\left(y\right)$. Since $y \in B$ arbitrary, we know that we must have $h = g$ and define $f^{-1} = h = g$.
\end{enumerate}
\end{proof}



\begin{problem}{1.2.6}
Let $f:\mathbb{R}\rightarrow \mathbb{R}$ be the function $f\left(x\right) = x^3 - x$. By restricting the domain and range of $f$ appropriately, obtain from $f$ a bijective function $g$. Draw the graphs of $g$ and $g^{-1}$. (There are several possible choices for $g$.)
\end{problem}


\subsection*{1.3 Relations}

\begin{problem}{1.3.1}
Define two points $\left(x_0,y_0\right)$ and $\left(x_1,y_1\right)$ of the plane to be equivalent if $y_0 - x_0^2 = y_1 - x_1^2$. Check that this is an equivalence relation and describe the equivalence classes.
\end{problem}
\begin{proof}
The relation is obviously reflexive, symmetric, and transitive merely by nature of the ordinary equality of the real numbers used in the defining equation of the relation. Thus, it is certainly an equivalence relation. The equivalence classes are of the form
\[ \left[ \left(x_0,y_0\right)\right] = \left\{ \left(x,y\right) \in \mathbb{R}^2 \, | \, y_0 - x_0^2 = y-x^2 \right\} \] Fix $\left(x_0,y_0\right)$ and let $y_0 - x_0^2 = r$. Then we can write \[ \left[ \left(x_0,y_0\right) \right] = \left\{ \left(x,y\right) \in \mathbb{R}^2 \, | \, y= x^2 + r \right\} \]
That is, this equivalence class is the set of points which all lie on the sample parabola $y=x^2 + r$ in the $\left(x,y\right)-$plane. Hence, the equivalence classes are all parabolas of the form $y = x^2 + k$ for $k \in \mathbb{R}$.
\end{proof}


\begin{problem}{1.3.2}
Let $C$ be a relation on a set $A$. If $A_0 \subset A$, define the \textbf{restriction} of $C$ to $A_0$ to be the relation $C\cap \left(A_0 \times A_0\right)$. Show that the restriction of an equivalence relation is an equivalence relation.
\end{problem}
\begin{proof}
Let $C$ be an equivalence relation on a set $A$ with some subset $A_0 \subset A$. Let $\tilde{C} = C\cap \left(A_0 \times A_0\right)$ be the restriction of $C$ to $A_0$. Let $x \in A_0$. Then $x \in A$ and so $\left(x,x\right) \in C$ since $C$ is reflexive as an equivalence relation. But $\left(x,x\right) \in A_0 \times A_0$ since $x \in A_0$; hence, $x \in \tilde{C}$, as it is the intersection of these two sets containing $x$. Hence, $\tilde{C}$ is reflexive. Next, let $\left(x,y\right) \in \tilde{C}$. Then $\left(x,y\right) \in C$ and so $\left(y,x\right) \in C$ since $C$ is an equivalence relation. Also, $\left(y,x\right) \in A_0 \times A_0$ since $x,y \in A_0$. Thus, $\left(y,x\right) \in \tilde{C}$ and so $\tilde{C}$ is symmetric.

Lastly, assume $\left(x,y\right), \left(y,z\right) \in \tilde{C}$. Then $x,y,z \in A_0$, so $\left(x,z\right) \in A_0\times A_0$. Also, we have $\left(x,z\right) \in C$ since $\left(x,y\right), \left(y,z\right) \in C$ by transitivity of equivalence relation $C$. Thus, $\left(x,z\right) \in \tilde{C}$ so $\tilde{C}$ is transitive. Finally, we have shown that the restriction of $C$ to a subset $A_0 \subset A$ is also an equivalence relation.
\end{proof}


\begin{problem}{1.3.3}
Here is a ``proof'' that every relation $C$ that is both symmetric and transitive is also reflexive: ``Since $C$ is symmetric, $aCb$ implies $bCa$. Since $C$ is transitive, $aCb$ and $bCa$ together imply $aCa$, as desired.'' Find the flaw in this argument.
\end{problem}
\begin{proof}
This proof would only work if for every $a$ there was some $b$ with $aCb$; on the contrary, it may be the case that there is some $a$ with no such $b$. In this case, we cannot say that a symmetric and transitive relation is reflexive.
\end{proof}


\begin{problem}{1.3.4}
Let $f:A\rightarrow B$ be a surjective function. Let us define a relation on $A$ by setting $a_0 \sim a_1$ if \[ f\left(a_0\right) = f\left(a_1\right). \]
\begin{enumerate}[label=(\alph*)]
	\item Show that this is an equivalence relation.
	\item Let $A^*$ be the set of equivalence classes. Show there is a bijective correspondence of $A^*$ with $B$.
\end{enumerate}
\end{problem}
\begin{proof}
\begin{enumerate}[label=(\alph*)]
	\item Clearly we have $f\left(a\right) = f\left(a\right)$ for every $a\in A$, so $\sim$ is reflexive. Let $x \sim y$. Then $f\left(x\right) = f\left(y\right)$. By symmetry of usual equality, we have $f\left(y\right) = f\left(x\right)$ and so $y\sim x$. Thus, $\sim$ is transitive. Lastly, let $x\sim y$ and $y\sim z$. Then $f\left(x\right) = f\left(y\right)$ and $f\left(y\right) = f\left(z\right)$, and by transitivity of usual equality, we have $f\left(x\right) = f\left(z\right)$, and so $x\sim z$, i.e., $\sim$ is transitive. Thus, $\sim$ is an equivalence relation.
	
	\item Let $A^*$ be the set of equivalence classes. Let $g$ be the function which sends the equivalence class $\left[ a\right] \in A^*$, where $a \in A$ is a representative element of the equivalence class, to the real value $f\left(a\right)$; that is, $g\left(\left[a\right]\right) = f\left(a\right)$. This is well-defined, since for any $x \in \left[a\right]$, $f\left(x\right) = f\left(a\right)$. We wish to show that $g$ is a bijection.
	
	First, let $\left[a\right], \left[b\right] \in A^*$ and assume $g\left(\left[a\right]\right) = g\left(\left[b\right]\right)$. Then $f\left(a\right) = f\left(b\right)$ and so $a$ and $b$ are in the same equivalence class, i.e., $\left[a\right]=\left[b\right]$. Thus, $g$ is injective. Next, assume that $ y \in B$. Then since $f$ is a surjective function, there is some $x \in A$ with $f\left(x\right) = y$. Then $g\left(\left[x\right]\right) = y$ and $g$ is surjective. Thus, $g$ is a bijection.
\end{enumerate}
\end{proof}



\begin{problem}{1.3.5}
Let $S$ and $S'$ be the following subsets of the plane:
\begin{equation*}
\begin{split} 
S &= \left\{ \left(x,y\right) \, | \, y = x+1 ~ \text{and} ~ 0 < x <2 \right\}, \\
S' & = \left\{ \left(x,y\right) \, | \, y-x ~\text{is an integer}\right\}.
\end{split}
\end{equation*}
\begin{enumerate}[label=(\alph*)]
	\item Show that $S'$ is an equivalence relation on the real line and $S' \supset S$. Describe the equivalence classes of $S'$.
	\item Show that given any collection of equivalence relations on a set $A$, their intersection is an equivalence relation on $A$.
	\item Describe the equivalence relation $T$ on the real line that is the intersection of all equivalence relations on the real line that contains $S$. Describe the equivalence classes of $T$.
\end{enumerate}
\end{problem}
\begin{proof}
\begin{enumerate}[label=(\alph*)]
	\item First, let $\left(x,y\right) \in S$, i.e., $y = x +1$ and $0 < x <2$. Then we have $y-x = 1$ and so $y-x$ is an integer (namely $1$). Thus, $\left(x,y\right) \in S'$, and so $S \subset S'$.
	
	To show that $S'$ is an equivalence relation, first note that $\left(x,x\right) \in S'$ since $x-x=0 \in \mathbb{Z}$, so $S'$ is reflexive. Next, let $\left(x,y\right) \in S'$. Then $x-y \in \mathbb{Z}$. But then $y-x = -\left(x-y\right) \in \mathbb{Z}$, so $\left(y,x\right) \in S'$, so $S'$ is symmetric. Lastly, let $\left(x,y\right), \left(y,z\right) \in S'$. Then $x-y$ and $y-z$ are integers. But then $x-z = x + \left(-y + y\right) - z = \left(x-y\right) + \left(y-z\right) \in \mathbb{Z}$ since it is the sum of two integers. Thus, $\left(x,z\right) \in S'$, and so $S'$ is transitive. Thus, $S'$ is an equivalence relation on the real number line. The equivalence classes of $S'$ are the subsets of $\mathbb{R}$ of the form \[ \left[x\right] = \left\{ \ldots, x-3, x-2, x-1, x, x+1, x+2, x+3, \ldots \right\} \] They are grids of points which are spaces 1 unit apart. 
	
	\item Let $S_i$ be an equivalence relation on a set $A$ for any $i\in I$ where $I$ is some indexing set, and define $S = \cap_{i \in I} S_i$. Let $x \in A$. Then $\left(x,x\right) \in S_i$ for each $i\in I$ since each $S_i$ is an equivalence relation. Thus $\left(x,x\right) \in S$, and so $S$ is reflexive. Next, let $\left(x,y\right) \in S$. Then $\left(x,y\right) \in S_i$ for each $i \in I$. Then by symmetry of each $S_i$, $\left(y,x\right) \in S_i$ for each $i \in I$ and so $\left(y,x\right) \in S$, i.e., $S$ is symmetric. Lastly, assume $\left(x,y\right),\left(y,z\right) \in S$. Then these are in $S_i$ for each $i\in I$. Then $\left(x,z\right) \in S_i$ for each $i \in I$, by transitivity of $S_i$. Then $\left(x,z\right) \in S$, so $S$ is transitive. Thus, $S$ is an equivalence relation on $A$.
	
	\item 
\end{enumerate}
\end{proof}



\begin{problem}{1.3.6}
Define a relation on the plane by setting \[ \left(x_0,y_0\right) < \left(x_1, y_1\right) \] if either $y_0-x_0^2 < y_1 - x_1^2$, or $y_0 - x_0^2 = y_1 - x_1^2$ and $x_0 < x_1$. Show that this is an order relation on the plane, and describe it geometrically.
\end{problem}
\begin{proof}
Let $\left(x_0,y_0\right), \left(x_1,y_1\right) \in \mathbb{R}^2$ distinct. By being distinct plane points, we either have $x_0 \neq x_1$ or $y_0 \neq y_1$ (or both). If $y_0 - x_0^2 < y_1 - x_1^2$ or $y_1 - x_1^2 < y_0 - x_0^2$ then the two are certainly comparable. Otherwise, if $y_0-x_0^2 = y_1-x_1^2$, then we must have $x_0 \neq x_1$; otherwise, $y_0 - x_0^2 = y_1 - x_1^2 \implies y_0 = y_1$ and then the two plane points are not distinct. Thus, any two points are comparable.

Next, assume that we have some plane point $\left(x,y\right)$ with $\left(x,y\right) < \left(x,y\right)$. Since $x = x$ and not $x < x$, we must have $y- x^2 < y-x^2$. This is clearly absurd. $\bot$ Thus this relation is nonreflexive.

Lastly, let $\left(x_0,y_0\right), \left(x_1,y_1\right), \left(x_2,y_2\right)$ be plane points with $\left(x_0,y_0\right) < \left(x_1,y_1\right)$ and $\left(x_1,y_1\right) < \left(x_2,y_2\right)$. By considering cases, we can show transitivity. If $y_0 - x_0^2 < y_1 - x_1^2$, and $y_1 - x_1^2 < y_2 - x_2^2$, then clearly $y_0 - x_0^2 < y_2 - x_2^2$ by transitivity of the standard order relation on the real line, and so $\left(x_0,y_0\right) < \left(x_2,y_2\right)$. Instead, if $y_1 - x_1^2 = y_2-x_2^2$, then we must have $x_1 < x_2$. But then $y_0 - x_0^2 < y_1 - x_1^2 = y_2 - x_2^2$ and so $\left(x_0,y_0\right) < \left(x_2,y_2\right)$. 
In another case, if $y_0 - x_0^2 = y_1 - x_1^2$, then $x_0 < x_1$. On one hand, if $y_1-x_1^2 < y_2 - x_2^2$, then $y_0-x_0^2 = y_1-x_1^2 < y_2 - x_2^2$ and so $\left(x_0,y_0\right) < \left(x_2,y_2\right)$. On the other hand, if $y_1 - x_1^2 = y_2 - x_2^2$, then $x_1 < x_2$, and we have $y_0 - x_0^2 = y_2 - y_2^2$ and $x_0 < x_1 < x_2$; thus we have $\left(x_0, y_0\right) < \left(x_2,y_2\right)$. 

In any case, we have the relation transitive. We have shown that this is an order relation on the plane. Consider two plane points $\left(x_0,y_0\right)$ and $\left(x_1,y_1\right)$. Then let $k_0 = y_0 - x_0^2$ and $k_1 = y_1 - x_1^2$. Then we have $\left(x_0,y_0\right) < \left(x_1,y_1\right)$ if $k_0 < k_1$ or if $k_0 = k_1$ and $x_0 < x_1$. If $k_0 < k_1$, then the two plane points lie on parabolas of the form $y=x^2 +k$ which one above the other. If $k_0 = k_1$, the plane points lie on the same parabola but one point lies to the right of the other (in the $x$ direction).
\end{proof}



\begin{problem}{1.3.7}
Show that the restriction of an order relation is an order relation.
\end{problem}
\begin{proof}
Let $<$ be an order relation on a set $A$. Let $A_0 \subset A$ and consider the restriction of $<$ to $A_0$, written as $<_{A_0}$. That is, $x<_{A_0} y$ if and only if $x<y$ and $x,y \in A_0$. Let $x,y,z \in A_0$. Then if $x<_{A_0} x$, then $x < x$ since $x\in A_0$. However, this is a contradiction by definition of the order relation $<$. $\bot$ Thus, $<_{A_0}$ is nonreflexive.

Next, assume $x \neq y$. Then assume $x \not<_{A_0} y$. Since $x,y \in A_0$, we must have $x \not< y$. Then since $x\neq y$, we must have $y < x$. Thus $y <_{A_0} x$. On the other hand, we have $x <_{A_0} y$. In either case, then, we have $x$ and $y$ comparable by the relation $<_{A_0}$.

Lastly, assume $x<_{A_0}y$ and $y <_{A_0} z$. Then since $x,y,z \in A_0$, we have $x<y$ and $y<z$, and by transitivity of the order relation $<$, we have $x<z$ and so $x <_{A_0} z$, and so $<_{A_0}$ is transitive.

Taken together, we have shown that the restriction $<_{A_0}$ of $<$ to $A_0 \subset A$ is itself an order relation.
\end{proof}


\begin{problem}{1.3.8}
Check that the relation defined in Example 7 is an order relation.
\end{problem}
\begin{proof}
The order relation in Example 7 is the relation on $\mathbb{R}$ consisting of all pairs $\left(x,y\right)$ such that $x^2 < y^2$, or if $x^2=y^2$ and $x<y$. If $x \neq y$, then if $x^2 < y^2$ or $y^2 < x^2$ then $x$ and $y$ are comparable by $<$. Otherwise, if $x^2 = y^2$, then since $x$ and $y$ are distinct, we must have $x = -y$, and so either $x < y$ or $y< x$ and so $x$ and $y$ are comparable again.

Next, assume that $x < x$ for some $x \in \mathbb{R}$. Certainly we do not have $x^2 < x^2$, so we must have $x<x$. This is nonsensical, and so $<$ is nonreflexive.

Lastly, let $x<y$ and $y<z$ for some $x,y,z \in \mathbb{R}$. We proceed in cases. In one case, we have $x^2 < y^2$. If $y^2 < z^2$, then $x^2 < z^2$ and so $x<z$. On the other hand, if $y^2 = z^2$, then $y<z$, and so $x^2 < y^2 = z^2$, so $x < z$. In the other case, we have $x^2 = y^2$ and so $x<y$. Then if $y^2 < z^2$, we have $x^2 = y^2 < z^2$ and so $x < z$. On the other hand, if $y^2=z^2$, then $y<z$ and so $x^2 = y^2 =z^2$ and so $x < z$. Thus, in any case, we have $<$ transitive.
\end{proof}




\begin{problem}{1.3.9}
Check that the dictionary order is an order relation.
\end{problem}
\begin{proof}
Recall that the dictionary order relation is the relation $<$ on a Cartesian product $A\times B$ defined by \[ a_1 \times b_1 < a_2 \times b_2 \] if $a_1 <_A a_2$, or if $a_1 = a_2$ and $b_1 <_B b_2$, where $<_A$ and $<_B$ are order relations on $A$ and $B$, respectively.

First, assume that $a\times b < a\times b$. Then certainly $a \not<_A a$, since $<_A$ is an order relation. Then since $a=a$, be must have $b <_B b$, but this is not true since $<_B$ is an order relation. $\bot$ Thus $<$ is nonreflexive.

Let $a_1 \times b_1$ and $a_2 \times b_2$ be two distinct points in $A\times B$. Then either $a_1 \neq a_2$ or $b_1 \neq b_2$ or both. If $a_1 \neq a_2$, then either $a_1 <_A a_2$ or $a_2 <_A a_1$. Then $ a_1\times b_1 < a_2 \times b_2 $ or $a_2 \times b_2 < a_1 \times b_1$, respectively; the points are comparable. On the other hand, if $a_1 = a_2$, then we must have $b_1 \neq b_2$, so either $b_1 <_B b_2$ or $b_2 <_B b_1$. Then $ a_1\times b_1 < a_2 \times b_2 $ or $a_2 \times b_2 < a_1 \times b_1$, respectively; the points are comparable. In either case, we have the distinct points comparable.

Lastly, let $a_1 \times b_1 < a_2 \times b_2$ and $a_2 \times b_2 < a_3 \times b_3$. Then we can consider cases and show that transitivity holds. I will not do this here because it is similar to what we did in the previous couple of problems, and very straightforward.

The dictionary order is an order relation.
\end{proof}



\begin{problem}{1.3.10}
\begin{enumerate}[label=(\alph*)]
	\item Show that the map $f:\left(-1,1\right)\rightarrow \mathbb{R}$ of Example 9 is order preserving.
	\item Show that the equation $g\left(y\right) = 2y/\left[1+\left(1+4y^2\right)^{1/2}\right]$ defines a function $g:\mathbb{R}\rightarrow \left(-1,1\right)$ that is both a left and a right inverse for $f$.
\end{enumerate}
\end{problem}
\begin{proof}
\begin{enumerate}[label=(\alph*)]
	\item The map in question is $f\left(x\right) = \frac{x}{1-x^2}$. Let $x < y$ for $x,y \in \left(-1,1\right)$. If $\left|x\right| < \left|y\right|$, then $x^2 < y^2$, and then $x^2-1 < y^2 -1$ and so $1-y^2 < 1-x^2$. Finally, $\frac{1}{1-x^2} < \frac{1}{1-y^2}$, and so $f\left(x\right) = \frac{x}{1-x^2} < \frac{x}{1-y^2} < \frac{y}{1-y^2} = f\left(y\right)$. On the other hand, if $\left|x\right| > \left|y\right|$, then $x^2 > y^2$. This is only possible along with $x<y$ if $x$ is negative but larger in magnitude than $y$, which is positive. Thus $f\left(x\right) =\frac{x}{1-x^2} < \frac{y}{1-y^2} = f\left(y\right)$ and since the denominators are positive numbers, and $x$ is negative while $y$ is positive. Hence, in any case we have $f\left(x\right) < f\left(y\right)$.
	
	\item We can clearly see that \[ -1 < \frac{2y}{1+\sqrt{1+4y^2}} < 1 \] We can show by direct computation that $f\circ g = i_{\left(-1,1\right)}$ and that $g \circ f = i_{\mathbb{R}}$.
\end{enumerate}
\end{proof}













\subsection*{1.4 The Integers and the Real Numbers}



\subsection*{1.5 Cartesian Products}



\subsection*{1.6 Finite Sets}



\subsection*{1.7 Countable and Uncountable Sets}



\subsection*{1.8 The Principle of Recursive Definition}



\subsection*{1.9 Infinite Sets and the Axiom of Choice}

\subsection*{1.10 Well-Ordered Sets}


\subsection*{1.11 The Maximum Principle}


\subsection*{1S. Well-Ordering}












\newpage
\section*{Chapter 2. Topological Spaces and Continuous Functions}

\begin{problem}{13.1}
Let $X$ be a topological space; let $A$ be a subset of $X$. Suppose that for each $x\in A$ there is an open set $U$ containing $x$ such that $U \subset A$. Show that $A$ is open in $X$.
\end{problem}
\begin{proof}
	For every $x\in A$, take $U_x$ to be the open set containing $x$ and inside $A$. Then take $U = \cup_{x\in A} U_x$ (note that this may even be an uncountable union). We certainly have $U$ open as an arbitrary union of open sets. Then we have $U=A$, and this is easy to see by our definition of $U$. Thus, $A$ is open.
\end{proof}

\begin{problem}{13.2}
	Consider the nine topologies on the set $X=\left\{a,b,c\right\}$ indicated in Example 1 of \S12. Compare them; that is, for each pair of topologies, determine whether they are compatible, and if so, which is the finer.
\end{problem}
\begin{proof}
It helps to write out the topologies as sets:
\begin{enumerate}[label=(\alph*)]
	\item $\mathcal{T}_a=\left\{ \varnothing, X \right\}$
	\item $\mathcal{T}_b=\left\{ \varnothing, X, \left\{a\right\}, \left\{a,b\right\} \right\}$
	\item $\mathcal{T}_c=\left\{ \varnothing, X, \left\{b\right\}, \left\{a,b\right\},\left\{b,c\right\} \right\}$
	\item $\mathcal{T}_d=\left\{ \varnothing, X, \left\{b\right\} \right\}$
	\item $\mathcal{T}_e=\left\{ \varnothing, X, \left\{a\right\}, \left\{b,c\right\} \right\}$
	\item $\mathcal{T}_f=\left\{ \varnothing, X, \left\{b\right\}, \left\{c\right\}, \left\{a,b\right\},\left\{b,c\right\} \right\}$
	\item $\mathcal{T}_g=\left\{ \varnothing, X, \left\{a,b\right\} \right\}$
	\item $\mathcal{T}_h=\left\{ \varnothing, X, \left\{a\right\}, \left\{b\right\}, \left\{a,b\right\} \right\}$
	\item $\mathcal{T}_i=\left\{ \varnothing, X, \left\{a\right\}, \left\{b\right\}, \left\{c\right\}, \left\{a,b\right\},\left\{a,c\right\}, \left\{b,c\right\} \right\} = 2^X$
\end{enumerate}
Clearly $\mathcal{T}_a$ is the indiscrete topology and is the coarsest possible topology on $X$.  Also of note, $\mathcal{T}_i$ is the discrete topology and is the finest possible topology. I will not go through the exercise of comparing the remaining topologies, but it is rather straightforward.
\end{proof}

\begin{problem}{13.3}
	Show that the collection $\mathcal{T}_c$ given in Example 4 of \S12 is a topology on the set $X$. Is the collection $$ \mathcal{T}_{\infty} = \left\{ U \, \middle| \, X\setminus U \text{ is infinite or empty or all of } X \right\} $$ a topology on $X$?
\end{problem}
\begin{proof}
	Recall that $$\mathcal{T}_c = \left\{ U\subset X \, \middle| \, X\setminus U \text{ is countable or is all of } X \right\}.$$ Certainly $X\setminus X=\varnothing$ is countable vacuously, and $X\setminus \varnothing = X$. Thus, $X,\varnothing \in \mathcal{T}_c$. Next, let $U_{\alpha} \in \mathcal{T}_c$ for all $\alpha$ in some indexing set $A$ (maybe uncountable). If any $U_{\alpha} = \varnothing$, then they contribute nothing to the union. If $U_{\alpha} = \varnothing$ for all $\alpha \in A$, then the union itself is $\varnothing$, whose complement is $X$, and so the union of the $U_{\alpha}$ is open. On the other hand, assume that at least one of the $U_{\alpha}$ is nonempty, say $U_{\beta}$. Then $$ X\setminus \cup_{\alpha \in A} U_{\alpha} = \cap_{\alpha \in A} X\setminus U_{\alpha} \subseteq X\setminus U_{\beta} $$ where the last set inclusion holds because we are taking an intersection. Since $U_{\beta}$ is nonempty, $X\setminus U_{\beta}$ is countable (if not finite); thus $X\setminus \cup_{\alpha \in A}U_{\alpha}$ is countable as a subset of $X\setminus U_{\beta}$, and so $\cup_{\alpha \in A}U_{\alpha} \in \mathcal{T}_c$.
	
	Last, let $U_{\alpha} \in \mathcal{T}_c$ for all $\alpha$ in some finite set $A$. Then if any $U_{\alpha} = \varnothing$, then $$ \cap_{\alpha \in A} U_{\alpha} = \varnothing, $$ which has complement all of $X$, and so this intersection is in $\mathcal{T}_c$. On the other hand, if all $U_{\alpha}$ are nonempty, then $$ X\setminus \cap_{\alpha \in A} U_{\alpha} = \cup_{\alpha \in A} X \setminus U_{\alpha} $$ and each of the elements in the finite --- and thus countable --- union  on the right-hand side is countable, so this is a countable union of countable sets, and is known to be countable. Thus, $\cap_{\alpha \in A} U_{\alpha} \in \mathcal{T}_c$. We have shown that $\mathcal{T}_c$ is indeed a topology on set $X$. 
	
	 
	
\end{proof}




\begin{problem}{13.4}
\begin{enumerate}[label=(\alph*)]
	\item If $\left\{\mathcal{T}_a\right\}$ is a family of topologies on $X$, show that $\cap \,\mathcal{T}_{\alpha}$ is a topology on $X$. Is $\cup\, \mathcal{T}_{\alpha}$ a topology on $X$?
	\item Let $\left\{\mathcal{T}_{\alpha}\right\}$ be a family of topologies on $X$. Show that there is a unique smallest topology on $X$ containing all the collections $\mathcal{T}_{\alpha}$, and a unique largest topology contained in all $\mathcal{T}_{\alpha}$.
	\item If $X=\left\{a,b,c\right\}$, let $$ \mathcal{T}_1=\left\{\varnothing, X, \left\{a\right\},\left\{a,b\right\}\right\} \text{   and   } \mathcal{T}_2=\left\{\varnothing, X, \left\{a\right\},\left\{b,c\right\}\right\}$$ Find the smallest topology containing $\mathcal{T}_1$ and $\mathcal{T}_2$, and the largest topology contained in $\mathcal{T}_1$ and $\mathcal{T}_2$.
\end{enumerate}
\end{problem}
\begin{proof}
\begin{enumerate}[label=(\alph*)]
	\item Certainly $\varnothing$ and $X$ are in $\cap \, \mathcal{T}_{\alpha}$ since they are necessarily in each $\mathcal{T}_{\alpha}$. Let $\left\{U_{\beta}\right\}_{\beta \in B}$ be an arbitrary collection of elements of $\cap \, \mathcal{T}_{\alpha}$. Then $\left\{U_{\beta}\right\}_{\beta \in B} \subseteq \mathcal{T}_{\alpha}$ for each $\alpha$. Thus, $\cup_{\beta \in B} U_{\beta} \in \mathcal{T}_{\alpha}$ for each $\alpha$, and so it is in the intersection $\cap\, \mathcal{T}_{\alpha}$. Similarly, assume $B$ is finite and then $\cap_{\beta \in B} U_{\beta} \in \mathcal{T}_{\alpha}$ for each $\alpha$, and so it is in $\cap\, \mathcal{T}_{\alpha}$. Hence, the intersection is a topology on $X$. {\color{red}It is also obviously the largest (finest) topology contained within all of the $\mathcal{T}_{\alpha}$.}
	
	On the other hand, the union of topologies is not, in general, a topology. For example, consider $X=\left\{a,b,c\right\}$ with topologies $\mathcal{T}_1 = \left\{\varnothing, X, \left\{a\right\}\right\}$ and $\mathcal{T}_2 = \left\{\varnothing, X, \left\{b\right\}\right\}$; then $\mathcal{T}_1 \cup \mathcal{T}_2 = \left\{\varnothing, X, \left\{a\right\}, \left\{b\right\}\right\}$ is not a topology, since $\left\{a\right\}\cup\left\{b\right\} = \left\{a,b\right\} $ is not open.
	
	\item In the previous part of this exercise, we claimed that the largest topology contained in all $\mathcal{T}_{\alpha}$ is the intersection $\cap \, \mathcal{T}_{\alpha}$. To show this, assume that there is some other topology $\mathcal{T}'$ which is contained in each $\mathcal{T}_{\alpha}$ and is larger than $\cap \,\mathcal{T}_{\alpha}$. Then there is some open set $U \in \mathcal{T}' $ but $U \notin \cap_{\alpha} \mathcal{T}_{\alpha}$. But then since $\mathcal{T}' \subset \mathcal{T}_{\alpha}$ for all $\alpha$, $U \in \cap_{\alpha} \mathcal{T}_{\alpha}$; this is a contradiction, and so $\cap \, \mathcal{T}_{\alpha}$. Hence the claim is proven. 
	
	Next, we want the smallest topology on $X$ containing all $\mathcal{T}_{\alpha}$. Note that the union $\cup \, \mathcal{T}_{\alpha}$ contains all of the topologies, but we know from the previous part that it is not generally a topology, since it is not guaranteed to be closed under arbitrary unions and finite intersections. With this in mind, we define $\mathcal{T}$ as the collection of all arbitrary unions and finite intersections of elements of $\cup \, \mathcal{T}_{\alpha}$. This is certainly a topology on $X$, and contains each of the $\mathcal{T}_{\alpha}$; why is it the smallest such topology? Assume there is some topology $\mathcal{T}'$ which contains each $\mathcal{T}_{\alpha}$ and is smaller than $\mathcal{T}$. Then there is some element $U\in \mathcal{T}$ with $U \notin \mathcal{T}'$. Then $U$ is either an arbitrary union or finite intersection of elements of $\cup \, \mathcal{T}_{\alpha}$. That means that this particular union or finite intersection of elements is missing from $\mathcal{T}'$, and so $\mathcal{T}'$ cannot be a true topology. Thus, our construction $\mathcal{T}$ is the smallest topology containing all $\mathcal{T}_{\alpha}$. {\color{red}Later, this is referred to as the topology generated by the subbasis $\left\{\mathcal{T}_{\alpha}\right\}$.}
	
	\item From the previous part of the problem, the smallest topology containing the two topologies is $$ \left\{ \varnothing, X, \left\{a\right\}, \left\{b\right\}, \left\{a,b\right\}, \left\{ a,b,c\right\}\right\} $$ and the largest topology contained in both is $$ \left\{ \varnothing, X, \left\{a\right\} \right\}. $$
\end{enumerate}
\end{proof}



\begin{problem}{13.5}
Show that if $\mathcal{A}$ is a basis for a topology on $X$, then the topology generated by $\mathcal{A}$ equals the intersection of all topologies on $X$ that contain $\mathcal{A}$. Prove the same if $\mathcal{A}$ is a subbasis.
\end{problem}

\begin{problem}{13.6}
Show that the topologies of $\mathbb{R}_l$ and $\mathbb{R}_K$ are not comparable.
\end{problem}
\begin{proof}
The first observation we make is that $\left[1,2\right)$, for example, is open in $\mathbb{R}_l$ but is not open in $\mathbb{R}_K$, since we cannot find an open interval or ``$K$-set'' about the point $1$. Thus,  $\mathbb{R}_l\not\subseteq \mathbb{R}_K$.

Conversely, notice that the set $\left(-1,1\right)-K$ is open in $\mathbb{R}_K$, as it is a basis element, but it is not open in $\mathbb{R}_l$ since $0$ is contained in the set, but any half-open interval containing $0$ necessarily contains some points greater than zero. In particular, let $\left[a,b\right)$ be a half-open interval containing $0$. Then $a\leq 0$, and $b>0$. But then there is some $n\in \mathbb{Z}_+$ so that $0<\frac{1}{n} < b$ (this is the Archimedean property of the integers). Then this point is in $K$, and so it is removed from the set $\left(-1,1\right)-K$; since it is present in $\left[a,b\right)$, we have $\left[a,b\right) \not\subseteq \left(-1,1\right)-K$. Hence, $\mathbb{R}_K \not\subseteq \mathbb{R}_l$.

Since neither topology is contained in the other, the two are not comparable.   
\end{proof}


\begin{problem}{13.7}
	Consider the following topologies on $\mathbb{R}$:

\end{problem}


\begin{problem}{13.8}
\begin{enumerate}[label=(\alph*)]
	\item Apply Lemma 13.2 to show that the countable collection $$ \mathcal{B} = \left\{\,\left(a,b\right)\, \middle| \, a<b,\, a \text{ and } b \text{ rational}\,\right\}$$ is a basis that generates the standard topology on $\mathbb{R}$.
	\item Show that the collection $$ \mathcal{C} = \left\{ \, \left[a,b\right) \, \middle| \, a<b, \, a \text{ and } b \text{ rational}\,\right\}$$ is a basis that generates a topology different from the lower limit topology on $\mathbb{R}$.
\end{enumerate}
\end{problem}
\begin{proof}
\begin{enumerate}[label=(\alph*)]
	\item Lemma 13.2 states that a collection of open sets $\mathcal{C}$ of $X$ is a basis for a given topology on $X$ if for every open set $U\subset X$ and $x\in U$ there is some $C \in \mathcal{C}$ so that $x \in C \subset U$. Hence, let $U$ be an arbitrary open set in $\mathbb{R}$ and let $x \in U$. Since we know the standard topology on $\mathbb{R}$ is generated by open intervals, there is some open interval, say $\left(c,d\right)$, so that $x \in \left(c,d\right) \subset U$. Next, since rationals are dense in the reals, there exist some $a,b\in \mathbb{Q}$ with $c < a < x$ and $x < b < d$, so that $x \in \left(a,b\right) \subset \left(c,d\right)\subset U$. Since $\left(a,b\right) \in \mathcal{B}$, the use of Lemma 13.2 completes the proof.
	
	\item The lower limit topology is generated by the basis of half-open intervals with real endpoints. Then the half-open intervals with real endpoints, themselves, are open sets. In particular, $\left[\sqrt{2},2\right)$ is open in this topology. However, this set is not open in the topology $\mathcal{C}$, since the point $\sqrt{2}\in \left[\sqrt{2},2\right)$ can not be enclosed within $\left[\sqrt{2},2\right)$ by a half-open interval with rational endpoints. Thus, the two topologies are not the same. {\color{red}In fact, $\mathcal{C}$ generates a subset of the lower limit topology.}
\end{enumerate}
\end{proof}







\begin{problem}{17.1}
Let $\mathcal{C}$ be a collection of subsets of the set $X$. Suppose that $\varnothing$ and $X$ are in $\mathcal{C}$, and that finite unions and arbitrary intersections of elements of $\mathcal{C}$ are in $\mathcal{C}$. Show that the collection $ \mathcal{T} = \left\{ \, X-C \, \middle| \, C \in \mathcal{C}\,\right\}$ is a topology on $X$.
\end{problem}
{\color{red}Note that if we have defined a topology via some open sets, the associated closed sets satisfy the above. This is going in the other direction: if we have a collection of ``closed'' sets, the complements constitute a topology.}\\

\begin{proof}
	The proof involves careful applications of DeMorgan's laws. First, $\varnothing, X \in \mathcal{T}$ since they are in $\mathcal{C}$ and $X-X=\varnothing$ and $X-\varnothing=X$.
	
	Next, let $\left\{ X-C_{\alpha}\right\}_{\alpha}$ be an arbitrary subset of $\mathcal{T}$. Then we have $$ \cup_{\alpha} \left(X-C_{\alpha}\right) = \cup_{\alpha} \left(X\cap C^c_{\alpha}\right) = X \cap \left( \cup_{\alpha}  C^c_{\alpha}\right). $$ Since $C_{\alpha}$ closed for every $\alpha$, $C_{\alpha}^c$ is open and the arbitrary union of these is open, by the axioms for a topology. Thus we have an arbitrary union of elements of $\mathcal{T}$ written as an intersection of two open sets, and thus open. So $\mathcal{T}$ is closed under arbitrary unions.
	
	Last, let $\left\{ X-C_{\alpha}\right\}_{\alpha}$ be an arbitrary finite subset of $\mathcal{T}$. Then $$ \cap_{\alpha} \left(X-C_{\alpha}\right) = \cap_{\alpha} \left( X\cap C_{\alpha}^c \right) = X \cap \left(\cap_{\alpha} C^c_{\alpha}\right). $$ Then since $C_{\alpha}$ closed, $C_{\alpha}^c$ open and so the finite intersection of these is open. Then we have written an arbitrary finite intersection of sets in $\mathcal{T}$ as an intersection of open sets, so $\mathcal{T}$ is closed under finite intersections. The topology axioms have been demonstrated.
\end{proof}


\begin{problem}{17.2}
Show that if $A$ is closed in $Y$ and $Y$ is closed in $X$, then $A$ is closed in $X$.
\end{problem}
\begin{proof}
Assume $A$ closed as a subset of $Y$. Then we know that $A$ can be written as $A=Y\cap C$ for some $C\subset X$ closed. Then if $Y$ also closed, we have $A$ as the (finite) intersection of closed sets in $X$, so $A$ is closed in $X$.
\end{proof}



\begin{problem}{17.4}
	Show that if $U$ is open in $X$ and $A$ is closed in $X$, then $U-A$ is open in $X$, and $A-U$ is closed in $X$.
\end{problem}
\begin{proof}
	Since $U$ open, $X-U$ is closed; since $A$ closed, $X-A$ is open. Then $U-A = U\cap \left(X-A\right)$, which is the intersection of open sets, and is open. Similarly, $A-U = A \cap \left(X-U\right)$ which is the intersection of closed sets, and is closed.
\end{proof}


\begin{problem}{17.5}
Let $X$ be an ordered set in the order topology. Show that $\overline{\left(a,b\right)}\subset \left[a,b\right]$. Under what conditions does equality hold?
\end{problem}
\begin{proof}
	Recall that in the order topology is generated by the basis of open intervals along with the half-open intervals containing the smallest and largest elements of $X$, if those exist. Let $\left(a,b\right)$ be some set in $X$. Then we can write $$ \left[a,b\right] = X-\left(\left(-\infty, a\right)\cup\left(b,\infty\right) \right) $$ where ``$\pm\infty$'' represents the smallest and largest values in $X$, if those exist. Then $\left[a,b\right]$ is clearly closed since it is the complement of the open set $\left(-\infty,a\right)\cup\left(b,\infty\right)$ (open since any point can be enclosed in an open interval). Since $\overline{\left(a,b\right)}$ is the intersection of all closed sets containing $\left(a,b\right)$, we have $\overline{\left(a,b\right)} \subset \left[a,b\right]$.
	
	Equality will hold if $\left[a,b\right] \subset \overline{\left(a,b\right)}$. Certainly we have $\left(a,b\right) \subset \overline{\left(a,b\right)}$, by definition of closure. Thus, equality holds if both $x=a$ and $x=b$ are limit points of $\left(a,b\right)$. This means that, for example, every open interval about $a$ must intersect $\left(a,b\right)$ at some point greater than $a$. This is equivalent to saying that $a$ has no immediate successor in $X$: if $a$ has immediate successor $y$, then we can take $\left(-\infty,y\right)$ as an open neighborhood of $a$ which does not intersect $\left(a,b\right)$ at any point other than $a$. Thus, $a$ and $b$ can have no immediate successor and predecessor, respectively.
\end{proof}


\begin{problem}{17.6}
Let $A$, $B$, and $A_{\alpha}$ denote subsets of a space $X$. Prove the following:
\begin{enumerate}[label=(\alph*)]
	\item If $A\subset B$, then $\overline{A} \subset \overline{B}$.
	\item $\overline{A\cup B} = \overline{A}\cup \overline{B}$.
	\item $ \overline{\cup \,A_{\alpha}} \supset \cup\, \overline{A_{\alpha}}$; give an example where equality fails.
\end{enumerate}
\end{problem}
\begin{proof}
\begin{enumerate}[label=(\alph*)]
	\item Let $A\subset B$. Then $\overline{B}$ is a closed set containing $B$, so it also contains $A$. Hence, since $\overline{A}$ is the intersection of all closed sets containing $A$, we have $\overline{A} \subset \overline{B}$.
	\item For $A,B\subset X$, we have $A\subset \overline{A}$ and $B\subset \overline{B}$, so that $A\cup B \subset \overline{A}\cup \overline{B}$, and $\overline{A}\cup\overline{B}$ is closed as a finite union of closed sets. Then for the same reason as in (\textit{a}), we have $\overline{A\cup B} \subset \overline{A}\cup\overline{B}$.
	
	Conversely, let $E$ be a closed set containing $A\cup B$. Then $E$ is a closed set containing $A$ and $B$, and so $\overline{A} \subset E$ and $\overline{B}\subset E$. 	
	\item This is the generalization of part (\textit{b}) to arbitrary (maybe uncountable) unions. The proof 
\end{enumerate}
\end{proof}




\begin{problem}{17.10}
Show that every order topology is Hausdorff.
\end{problem}
\begin{proof}
Let $X$ be a set with the order topology, and choose $x,y \in X$, distinct. Assume without loss of generality that $x < y$, which is possible since the set is ordered. Then if there is some element $z$ between $x$ and $y$, we can take $U=\left(-\infty,z\right)$ and $V=\left(z,\infty\right)$ as disjoint open neighborhoods of $x$ and $y$, respectively.

On the other hand, if $y$ is the immediate successor of $x$, we can take $U=\left(-\infty,y\right)$ and $V=\left(x,\infty\right)$ as similar disjoint open neighborhoods of $x$ and $y$, respectively, so that $X$ is Hausdorff.
\end{proof}


\begin{problem}{17.11}
Show that the product of two Hausdorff spaces is Hausdorff.
\end{problem}
\begin{proof}
Let $\left(X,\mathcal{T}_X\right)$ and $\left(Y,\mathcal{T}_Y\right)$ be two Hausdorff spaces, and let $X\times Y$ be the associated product topological space. Then given distinct points $\left(x,y\right), \left(s,t\right) \in X\times Y$, we have either $x\neq s$, $y\neq t$, or both. First, assume that $x\neq s$ but $y=t$. Then since $X$ is Hausdorff, we have disjoint open sets $U,V\subset X$ so that $x\in U$ and $s\in V$; then $\left(x,y\right) \subset U\times Y$ and $\left(s,t\right)\subset V\times Y$ which are disjoint open sets in the product topology (basis elements). The case of $x=s$ but $y\neq t$ is treated similarly. 

Lastly, if $x\neq s$ and $y\neq t$, then we have open sets $U_x, V_x \subset X$ and $U_y,V_y \subset Y$ with $x \in U_x, s\in V_x, U_x \cap V_x = \varnothing$ and $y \in U_y, t\in V_y, U_y\cap V_y = \varnothing$. Then $\left(x,y\right) \in U_x \times U_y $ and $\left(s,t\right) \in V_x \times V_y $ are disjoint open sets in the product topology. Thus, in any case we have the product topology Hausdorff.
\end{proof}


\begin{problem}{17.12}
Show that a subspace of a Hausdorff space is Hausdorff.
\end{problem}
\begin{proof}
Let $X$ be a Hausdorff space with subspace $Y$. Then let $x,y \in Y$. Then since $Y\subset X$, there are open sets $U,V\subset X$ with $x\in U, y\in V$, and $U \cap V = \varnothing$. Then the sets $Y\cap U$ and $Y\cap V$ are open in the subspace topology on $Y$ and are disjoint open neighborhoods of $x$ and $y$, respectively, in $Y$. Hence, $Y$ is also Hausdorff. 
\end{proof}


\begin{problem}{17.13}
Show that $X$ is Hausdorff if and only if the diagonal $\Delta = \left\{\, x\times x\, \middle| \, x\in X \, \right\}$ is closed in $X \times X$.
\end{problem}
\begin{proof}
Assume $X$ is Hausdorff and let $y \times z \notin \Delta$, i.e., $y\neq z$ for $y,z \in X$. Then there are open sets $U,V \subset X$ with $y\in U, z\in V,$ and $U\cap V = \varnothing$. Then there is no $s\in X$ in both $U$ and $V$. Hence $y \times Z \subset U\times V$, open, and $\left(U\times V \right)\cap \Delta = 0$.

Conversely, if $\Delta$ is closed, then for any $y\times z \notin \Delta$, there is an open neighborhood $W$ of $y\times z$ which does not intersect $\Delta$, so $x \times x \in W$ for no $x \in X$. Then since $W$ open, we have by definition of the product topology open sets $U,V\subset X$ with $y \in U$, $z \in V$. Then $U $ and $V$ are disjoint since if $x \in U\cap V$, then $x\times x \in W$ and we already ruled this out. Thus, we have separated any arbitrary distinct $y,z \in X$ by disjoint open sets.
\end{proof}



\begin{problem}{17.14}
	In the finite complement topology on $\mathbb{R}$, to what point or points does the sequence $x_n = 1/n$ converge?
\end{problem}
\begin{proof}
	Recall that a sequence $\left(x_n\right)$ converges to limit $x$ if every open neighborhood of $x$ contains all $\left(x_n\right)$ for $n>N$ for some $N\in \mathbb{N}$. This depends on the definition of open sets in our space. In the finite complement topology, the open sets are those with finite complement, or complement which is all of $X$.
	
	With this in mind, let $x$ be an arbitrary point of $\mathbb{R}$ and let $U$ be an open set containing $x$. Then $U$ has finite complement, so a finite number of the $x_n$ are in this complement, i.e., $U$ contains all but finitely many elements of the sequence. Hence, we have some $N$ for which $x_n \in U$ for $n>N$, i.e., $\left(x_n\right)$ converges to every point in $\mathbb{R}$ in the finite complement topology.
\end{proof}


\begin{problem}{17.15}
	Show the $T_1$ axiom is equivalent to the condition that for each pair of points of $X$, each has a neighborhood not containing the other.
\end{problem}
\begin{proof}
	Let $X$ be $T_1$, i.e., finite point subsets of $X$ are closed, and let $x,y \in X$ be distinct points. Then, in particular, $\left\{x\right\}$ and $\left\{y\right\}$ are closed. Then $X\setminus \left\{x\right\}$ is an open set containing $y$ and not $x$, and $X\setminus \left\{x\right\}$ is an open set containing $x$ and not $y$.
	
	Conversely, assume that for each pair of points of $X$, each has a neighborhood not containing the other. Let $F= \left\{x_1,x_2,\ldots,x_n\right\}$ be a finite point set in $X$. Then let $y \notin F$. Then by our hypothesis, for each $x_i\in F$ we have some neighborhood $U_{i}$ containing $y$ which does not contain $x_i$. Then $U = \cap_{i=1}^n U_i$ is an open set containing $y$ which does not intersect $F$, so $F$ is closed.
\end{proof}



\subsection*{Continuous Functions}

\begin{problem}{18.1}
	Prove that for functions $f:\mathbb{R}\rightarrow \mathbb{R}$, the $\epsilon$-$\delta$ definition of continuity implies the open set definition.
\end{problem}
\begin{proof}
	Let $f:\mathbb{R}\rightarrow \mathbb{R}$ be continuous according to the $\epsilon$-$\delta$ definition and let $V\subset f\left(X\right)$ be some open set. Then let $y\in V$ be arbitrary, and consider $x \in f^{-1}\left(V\right)$. Then $f\left(x\right)=y \in V$, and so there is some basis element $\left(y-\epsilon,y+\epsilon\right) \subset V$ containing $y$ (we can take this interval to be symmetric by restricting $\epsilon$ to be small enough). Then by continuity of $f$, we have some $\delta >0$ with $f\left(\left(x-\delta,x+\delta\right)\right) \subset \left(y-\epsilon,y+\epsilon\right) \subset V$; that is, $\left(x-\delta, x+\delta\right) \subset f^{-1}\left(V\right)$. This completes the proof: $f$ is continuous in the topological space sense.
\end{proof}


\begin{problem}{18.2}
	Suppose that $f:X\rightarrow Y$ is continuous. If $x$ is a limit point of the subset $A$ of $X$, is it necessarily true that $f\left(x\right)$ is a limit point of $f\left(A\right)$?
\end{problem}
\begin{proof}
	This is not true. For example, the constant function $f\left(x\right)=y_0$ for all $x\in X$ is certainly continuous, but no open neighborhood of $\left\{y_0\right\}$ in $Y$ will contain any other point of $\left\{y_0\right\}$ (there is no other point).
\end{proof}


\begin{problem}{18.3}
Let $X$ and $X'$ denote a single set in the two topologies $\mathcal{T}$ and $\mathcal{T}'$, respectively. Let $i:X'\rightarrow X$ be the identity function.
\begin{enumerate}[label=(\alph*)]
	\item Show that $i$ is continuous $\iff$ $\mathcal{T}'$ is finer than $\mathcal{T}$.
	\item Show that $i$ is a homeomorphism $\iff$ $\mathcal{T}'=\mathcal{T}$.
\end{enumerate}
\end{problem}
\begin{proof}
\begin{enumerate}[label=(\alph*)]
	\item Assume $i$ is continuous. Then $U \in \mathcal{T} \implies i^{-1}\left(U\right)=U \in \mathcal{T}'$, so $\mathcal{T}\subseteq \mathcal{T}'$. Conversely, if $\mathcal{T}'\supseteq \mathcal{T}$, then $U \in \mathcal{T}' \implies U \in \mathcal{T}$, so that $i^{-1}\left(U\right) = U \in \mathcal{T}$, hence $i$ is continuous.
	\item This is obvious from part (\textit{a}) applied to both $i$ and $i^{-1}$.
\end{enumerate}
\end{proof}



\begin{problem}{18.4}
Given $x_0 \in X$ and $y_0\in Y$, show that the maps $f:X\rightarrow X\times Y$ and $g:Y\rightarrow X\times Y$ defined by $$ f\left(x\right) = x\times y_0 \text{   and   } g\left(y\right) = x_0 \times y $$ are imbeddings.
\end{problem}
\begin{proof}
Recall that a topological imbedding is a function which is a homeomorphism onto its image. Consider the function $f$ defined above. Then the image of $f$ is clearly $f\left(X\right) = X \times \left\{y_0\right\} \subset X \times Y$. Let $f\left(x_1\right) = f\left(x_2\right)$; then $x_1 \times y_0 = x_2 \times y_0$ which is true if and only if $x_1 = x_2$, by definition of Cartesian products. Thus, $f$ is injective. Next, let $z \in X \times \left\{y_0\right\}$ be an arbitrary element in the image of $X$ under $f$. Then again by definition of Cartesian product, we have $z = x \times y_0$ for some $x \in X$, and so we have $f\left(x\right) = x\times y_0 = z$, and so $f$ is also surjective. Hence, $f$ is a bijection, and has an inverse defined on $f\left(X\right)$.

We must show that $f$ and its inverse $f^{-1}$ are both continuous. Let $U$ be open in $X\times \left\{y_0\right\}$ and pick $z \in f^{-1}\left(U\right)$. Then $f\left(z\right) \in U \subset X\times \left\{y_0\right\}$, so $z= x\times y_0$ for some $x\in X$. Since $U$ open, there is some basis element $B_x \times B_y$ about $z$ which is contained in $U$, and where $B_x$ open in $X$ and $B_y$ open in $Y$ (this is by definition of the basis for the product topology). But then we must have $f^{-1}\left(B_x \times B_y\right) = B_x \subset f^{-1}\left(U\right)$, open, and $x \in B_x$. Hence, we have shown that $f$ is continuous. Analogously, $g$ is continuous as well.

Last, the inverse of $f$ is $f^{-1}\left(x\times y_0\right) = x$ which can be written as $f^{-1} = \pi_1$, the projection function onto the first component, which we know to be a continuous function on any product space. Similarly, $g^{-1} = \pi_2$ is continuous. Thus, both $f$ and $f^{-1}$ are continuous as maps between $X$ and $f\left(X\right)$, and so $f$ (and $g$) are topological imbeddings.
\end{proof}



\begin{problem}{18.5}
Show that the subspace $\left(a,b\right)$ of $\mathbb{R}$ is homeomorphic with $\left(0,1\right)$ and the subspace $\left[a,b\right]$ of $\mathbb{R}$ is homeomorphic with $\left[0,1\right]$.
\end{problem}
\begin{proof}
	We can easily provide the homeomorphism $f:\left(a,b\right) \rightarrow \left(0,1\right)$ as $$ f\left(x\right) = \frac{x-a}{b-a} $$ which has inverse $$ f^{-1}\left(y\right) = a + y\cdot (b-a), $$ both clearly continuous as linear functions in the $\epsilon$-$\delta$ definition. The same functions work in the second case, extended easily to the domains inclusive of the endpoints.
\end{proof}

\begin{problem}{18.6}
	Find a function $f:\mathbb{R}\rightarrow \mathbb{R}$ that is continuous at precisely one point.
\end{problem}
\begin{proof}
	From some minor familiarity with analysis, an example that springs to mind is the function $$f\left(x\right) = \left\{ \begin{array}{rl}
	x^2, & x \text{ rational} \\
	0, & x \text{ not rational}
\end{array}	 \right. $$
Consider $x \neq 0$. Then if $x$ rational, $f\left(x\right)=x^2 >0 $ and so we can take $\epsilon$ so that $0 < \epsilon < x^2$. But then any neighborhood $\left(x-\delta,x+\delta\right)$ will contain an irrational number $x'$ so that $\left|x'-x\right|<\delta$ but $\left| f\left(x'\right) -f\left(x\right)\right| = x^2 > \epsilon$. Since this holds for any $\delta$ we choose, $f$ cannot be continuous at any nonzero rational $x$. On the other hand, we can similarly show that $x$ cannot be continuous at any nonzero irrational $x$.

At $x=0$, we have $f\left(x\right)=0$, and given $\epsilon>0$ we may take $\delta = \epsilon^{1/2}$ so that $$ \left|x'-0\right| < \epsilon^{1/2} \implies \left|f\left(x'\right) - f\left(0\right)\right| = \left|f\left(x'\right)\right| < \epsilon $$ so that $f$ is continuous at $0$, which is only a single point.
\end{proof}


\begin{problem}{18.7}
\begin{enumerate}[label=(\alph*)]
	\item Suppose that $f:\mathbb{R}\rightarrow \mathbb{R}$ is ``continuous from the right,'' that is, $$ \lim_{x\rightarrow a^+} f\left(x\right) = f\left(a\right), $$ for each $a\in \mathbb{R}$. Show that $f$ is continuous when considered as a function from $\mathbb{R}_l$ to $\mathbb{R}$.
	\item Can you conjecture what functions $f:\mathbb{R}\rightarrow \mathbb{R}$ are continuous when considered as maps from $\mathbb{R}$ to $\mathbb{R}_l$? As maps from $\mathbb{R}_l$ to $\mathbb{R}_l$? We shall return to this question in Chapter 3.
\end{enumerate}
\end{problem}
\begin{proof}
\begin{enumerate}[label=(\alph*)]
	\item Let $U\in \mathbb{R}$ be an open set and consider arbitrary $x \in f^{-1}\left(U\right)$. Then $f\left(x\right) \in U$ and so there is some basis element $\left(f\left(x\right)-\epsilon, f\left(x\right)+\epsilon\right) \subset U$ containing $f\left(x\right)$. Recall that the limit hypothesis is that for any $\epsilon>0$ we have some $\delta>0$ with $$ 0\leq x'-x < \delta \implies \left|f\left(x'\right)-f\left(x\right)\right|<\epsilon. $$ Then since $f$ is right-continuous at $x$, we have a $\delta>0$ with $x\leq x' < x'+\delta \implies f\left(x'\right) \in \left(f\left(x\right)-\epsilon, f\left(x\right)+\epsilon\right) \subset U$, i.e., $\left[x,x+\delta\right)\subset f^{-1}\left(U\right)$, which is open in $\mathbb{R}_l$. Hence, $f$ is continuous in the topological sense. The converse is also very clearly true.
	
	\item 
\end{enumerate}
\end{proof}



\begin{problem}{18.8}
	Let $Y$ be an ordered set in the order topology. Let $f,g:X\rightarrow Y$ be continuous.
	\begin{enumerate}[label=(\alph*)]
		\item Show that the set $\left\{ x\, \middle| \, f\left(x\right)\leq g\left(x\right)\right\}$ is closed in $X$.
		\item Let $h:X\rightarrow Y$ be the function $$ h\left(x\right) = \min \left\{f\left(x\right),g\left(x\right) \right\}.$$ Show that $h$ is continuous. [\textit{Hint}: Use the pasting lemma.]
	\end{enumerate}
\end{problem}
\begin{proof}
		\begin{enumerate}[label=(\alph*)]
			\item Let $x$ not be in the set so that $f\left(x\right) > g\left(x\right)$. We can write this differently as $h\left(x\right)>0$ where $h:=f-g$ is clearly continuous. Then since $h$ continuous at $x$, there is some open neighborhood $U\subset X$ of $x$ so that $h>0$ on $U$ (this is a classic theorem in analysis and is easy to prove via the limit definition of continuity). Then the set $U$ is a neighborhood of $x$ which does not intersect the set of interest, and so that set is closed in $X$.
			\item We can write the function $h$ as $$ h\left(x\right) = \left\{ \begin{array}{rl} f\left(x\right), & f\left(x\right) \geq g\left(x\right) \\ g\left(x\right), & g\left(x\right) \geq f\left(x\right) \end{array}\right. .$$ Letting $A=\left\{ x \in X \, \middle| \, f\left(x\right) \geq g\left(x\right)\right\}$ and $B=\left\{ x \in X \, \middle| \, g\left(x\right) \geq f\left(x\right)\right\}$, we have $f|A:A\rightarrow Y$ and $g|B:B\rightarrow Y$ continuous as continuous functions on restricted domains; $f$ and $g$ certainly agree on the intersection, which is the set of points where $f\left(x\right) = g\left(x\right)$. Hence, the pasting lemma applies and so $h$ as defined is continuous.
		\end{enumerate}
\end{proof}





\begin{problem}{18.9}
	Let $\left\{A_{\alpha}\right\}$ be a collection of subsets of $X$; let $X=\cup_{\alpha}A_{\alpha}$. Let $f:X\rightarrow Y$; suppose that $f|A_{\alpha}$ is continuous for each $\alpha$.
	\begin{enumerate}[label=(\alph*)]
		\item Show that if the collection $\left\{A_{\alpha}\right\}$ is finite and each set $A_{\alpha}$ is closed, then $f$ is continuous.
		\item Find an example where the collection $\left\{A_{\alpha}\right\}$ is countable and each $A_{\alpha}$ is closed, but $f$ is not continuous.
	\end{enumerate}
\end{problem}
\begin{proof}
	\begin{enumerate}[label=(\alph*)]
		\item Let $V\subset Y$ be closed and consider $f^{-1}\left(V\right) \subset X$. Then we have $f^{-1}\left(V\right) = \cup_{\alpha \in A} \left(  f^{-1}\left(V\right) \cap A_{\alpha} \right)$ since $X$ has this decomposition. But notice that $f^{-1}\left(V\right)\cap A_{\alpha} = \left( f|A_{\alpha}\right)^{-1}\left(V\right)$. Thus, each $f^{-1}\left(V\right) \cap A_{\alpha} $ is closed because it is the preimage of a closed set under a continuous function. Then $f^{-1}\left(V\right)$ is a finite union of closed sets, so it is closed, and so $f$ is continuous.
		\item 
	\end{enumerate}	
\end{proof}

\begin{problem}{18.10}
	Let $f:A\rightarrow B$ and $g:C\rightarrow D$ be continuous functions. Let us define a map $f\times g: A\times C \rightarrow B\times D$ by the equation $$ \left(f\times g\right)\left(a\times c\right) = f\left(a\right)\times g\left(c\right).$$ Show that $f\times g$ is continuous.
\end{problem}
{\color{red}Note that this is not exactly the situation in Theorem 18.4; in that case, both $f$ and $g$ have the same domain, and so the function is closer to a vector-valued function. Here, the functions take different domains.}\\

\begin{proof}
	Let $U$ be an open subset of $B\times D$ and consider $\left(f\times g\right)^{-1}\left(U\right)$. To show that $f\times g$ is continuous, let $x \times y \in \left(f\times g\right)^{-1}\left(U\right) \subseteq A\times C$. Then $\left(f\times g\right)\left(x\times y\right) \in U\subset B\times D$, and since $U$ is open, there is some basis element, say $B_B \times B_D$, where $B_B$ open in $B$ and $B_D$ open in $D$, so that $$ \left(f\times g\right)\left(x\times y\right) \in B_B \times B_D \subset U.$$ Then $x \in f^{-1}\left(B_B\right)$ and $y \in g^{-1}\left(B_D\right)$ and both are open in $A$ and $C$, respectively, since $f$ and $g$ are continuous. Then there are basis elements $B_A$ and $B_C$ of $A$ and $C$, respectively, so that $$ x \in B_A \subset f^{-1}\left(B_B\right)$$ and $$ y \in B_C \subset g^{-1}\left(B_D\right). $$
	
	
	Then we have $$ x\times y \in B_A \times B_C \subset f^{-1}\left(B_B\right) \times g^{-1}\left(B_D\right) \subset \left(f\times g\right)^{-1}\left(U\right) $$ so that $\left(f\times g\right)^{-1}\left(U\right)$ is open, and so $f\times g$ is continuous.
\end{proof}\\

{\color{red}The following is an alternate, more elegant proof. We eschew open sets in favor of basis elements, since we can always find some basis elements in any open sets. The proof is the same as the one above, but with most detail left to the imagination (and previous experience).}\\

\begin{proof}
Let $U\times V$ be a basis element of $B\times D$. Then we have $$ \left(f\times g\right)^{-1}\left(U\times V\right) = f^{-1}\left(U\right)\times g^{-1}\left(V\right)$$ by definition of $f\times g$. This is a product of open sets, since $f$ and $g$ are continuous, and so is open in the product topology on $A\times C$. Since the preimage of any basis element is open, the function is continuous.
\end{proof}





\begin{problem}{18.11}
	Let $F:X\times Y \rightarrow Z$. We say that $F$ is continuous in each variable separately if for each $y_0 \in Y$, the map $h:X\rightarrow Z$ defined by $h\left(x\right) = F\left(x\times y_0\right)$ is continuous, and for each $x_0$ in $X$, the map $k:Y\rightarrow Z$ defined by $k\left(y\right) = F\left(x_0 \times y\right)$ is continuous. Show that if $F$ is continuous, then $F$ is continuous in each variable separately.
\end{problem}
\begin{proof}
	Let $U \subset Z$ be open and consider $x \in h^{-1}\left(U\right)$. Then $h\left(x\right) = F\left(x\times y_0\right) \in U$, and so there is a basis set $V$ so that $F\left(x\times y_0\right) \in V \subset U$. By continuity of $F$, $F^{-1}\left(V\right)$ open in $X\times Y$ and contains the point $x\times y_0$. Then there are basis elements $U_X$ and $U_Y$ so that $x\times y_0 \in U_X \times U_Y \subset F^{-1}\left(V\right)$. In particular, $x \in U_X$ and $U_X \subset h^{-1}\left(U\right)$. Why? Because if $x' \in U_X$, then $x' \times y_0 \in F^{-1}\left(V\right)$ so that $F\left(x' \times y_0\right) = h\left(x'\right) \in V \subset U$. Hence we have an open set about $x$ and in $h^{-1}\left(U\right)$, so $h$ is continuous. The function $k$ is continuous by symmetry.
\end{proof}






\begin{problem}{18.12}
Let $F:\mathbb{R}\times \mathbb{R} \rightarrow \mathbb{R}$ be defined by the equation $$ F\left(x\times y\right) = \left\{ \begin{array}{ll}
xy/\left(x^2+y^2\right) & \text{if }x\times y \neq 0\times 0.\\
0 & \text{if }x\times y = 0 \times 0.
\end{array}\right. $$	
\begin{enumerate}[label=(\alph*)]
	\item Show that $F$ is continuous in each variable separately.
	\item Compute the function $g:\mathbb{R}\rightarrow \mathbb{R}$ defined by $g\left(x\right)=F\left(x\times x\right)$.
	\item Show that $F$ is not continuous.
\end{enumerate}
\end{problem}
\begin{proof}
\begin{enumerate}[label=(\alph*)]
	\item First, fix $y=y_0$ and consider $f_1\left(x\right) = F\left(x\times y_0\right)$. if $y_0=0$, then $f_1$ is identically zero and so is continuous as a constant function. On the other hand, assume $y_0\neq 0$. Then when $x\neq 0$, we have $f_1\left(x\right) = x y_0 / \left(x^2 + y_0^2\right)$ which is continuous as the ratio of continuous functions (polynomials in $x$). Then since $$ \left| f_1\left(x\right) - 0\right| = \left| \frac{xy_0}{x^2+y_0^2} \right| < \left| x \right| \left| y_0\right|^{-1} $$ which can certainly be made as small as possibly by judicious choice of a bound on $x$, i.e., $f_1$ is continuous at zero. Hence, $F$ is continuous in the first variable. By symmetry, we can easily see that $F$ is continuous in both variables, separately.
	\item Define $g\left(x\right)=F\left(x\times x\right)$. It is straightforward to compute that $$ g\left(x\right) = \left\{ \begin{array}{ll}
	\frac{1}{2} & \text{if }x\neq 0 \\
	0 & \text{if }x = 0 \\
	\end{array}\right. $$
	\item It is obvious that $g$ has a point discontinuity at $x=0$. To prove it, let $\epsilon = \frac{1}{4}$ and notice that any neighborhood of $x=0$ will contain some nonzero points. More specifically, we have found an $\epsilon>0$ and an $x$ so that $$ x' \in \left(-\delta,\delta\right)  \text{ but } \left|g\left(x'\right)-g\left(0\right)\right| = \frac{1}{2} > \frac{1}{4}  $$ for any $\delta>0$. Hence, $g$ discontinuous at $x=0$. 
	
	{\color{red}A more topological argument would be that the set $\left\{\frac{1}{2}\right\}$ is closed in the range, but $F^{-1}\left(\left\{\frac{1}{2}\right\}\right)= \left\{ x\times x\, \middle| \, x \neq 0 \right\}$ is not closed in $\mathbb{R}\times \mathbb{R}$.} 
\end{enumerate}
\end{proof}


\begin{problem}{18.13}
Let $A\subset X$; let $f:A\rightarrow Y$ be continuous; let $Y$ be Hausdorff. Show that if $f$ may be extended to a continuous function $g:\overline{A}\rightarrow Y$, then $g$ is uniquely determined by $f$.
\end{problem}
{\color{red}Note that this includes the special case where $\overline{A}=X$ for some set $X$, i.e., where $A$ is dense in $X$. This case states that if a continuous function on a dense subset can be continuously extended to the closure, then the function is uniquely determined by its values on that dense subset.

My original idea for the proof involved taking any limit point and obtaining a sequence of points limiting toward that limit point and showing that the value of $g$ must be uniquely determined by such a sequence, but this doesn't quite hold since we don't know that such a sequence exists. In a metric space, we could construct it.}\\

\begin{proof}
	Assume that $g$ and $h$ disagree on some point $x \in \overline{A}$. We must have $x$ a limit point of $A$, since otherwise $x\in A$ and $g$ and $h$ certainly agree there. Then since $g\left(x\right)\neq h\left(x\right)$ and $Y$ is Hausdorff, we can find open neighborhoods $U$ and $V$ of $g\left(x\right)$ and $h\left(x\right)$, respectively, which do not intersect. Then the preimage $g^{-1}\left(U\right)$ and $h^{-1}\left(V\right)$ are both open neighborhoods of $x$ since $g,h$ are continuous. Then $g^{-1}\left(U\right)\cap h^{-1}\left(V\right)$ is also an open neighborhood of $x$. Then since $x$ is a limit point of $A$, there is some point $a\in A$ in this intersection, i.e., $g\left(a\right)=h\left(a\right)$. But $g\left(a\right) \in U$ and $h\left(a\right) \in V$, assumed disjoint. This is a contradiction.
\end{proof}\\


\begin{proof}
	Let $x$ be an arbitrary point in $\overline{A}$. Assume that there are functions $g,h:\overline{A}\rightarrow Y$ which are continuous extensions of $f$. Then set $$ S = \left\{ x\in \overline{A} \, \middle| \, g\left(x\right)=h\left(x\right) \right\}. $$ By definition, $S \subseteq \overline{A}$, and $A \subseteq S$ since $g$ and $h$ must agree on $A$; put concisely, $$ A \subseteq S \subseteq \overline{A}.$$ The goal is to show that $S$ is exactly $\overline{A}$.
	
	Note that we can write $S$ as the diagonal set $$ \Delta = \left\{ \left(y,y\right) \, \middle| \, y \in Y\right\}$$ Think of $g$ and $h$ as coordinates in $Y$, written as the function $$ \phi\left(x\right) = \left(g\left(x\right),h\left(x\right)\right) \in Y $$ Then $$ S = \phi^{-1}\left(\Delta\right).$$ We know from a previous exercise that $\Delta$ is a closed set since $Y$ is a Hausdorff space, so $\overline{\Delta}=\Delta$; we also know from Theorem 18.4 that $\phi$ is continuous, so that $\phi^{-1}\left(\Delta\right)=S$ is closed, hence $\overline{S}=S$. But then $A \subseteq S \subseteq \overline{S}$, and so $\overline{S}$ is a closed set containing $A$, and $\overline{A}$ is the intersection of all such sets, so $\overline{A}\subseteq \overline{S}=S$. Hence, we have shown that $\overline{A}=S$, so that $h$ and $g$ agree on all of $\overline{A}$.
\end{proof}






\begin{problem}{19.4}
	Show that $\left(X_1\times \cdots \times X_{n-1}\right) \times X_n$ is homeomorphic with $X_1 \times \cdots \times X_n$.
\end{problem}
{\color{red}This statement is important because it asserts that parenthesis can be ignored from the topological aspect. Note that this proof is only carried out for finite Cartesian product spaces, and that in these cases the box and the product topology are the same.}\\

\begin{proof}
	It is possible to build this homeomorphism by looking at the elements of the product spaces as functions, but it is not so enlightening. Instead, we treat them as tuples to obtain a cleaner proof. The homeomorphism is $\phi: \left(X_1\times \cdots \times X_{n-1}\right)\times X_n \rightarrow X_1 \times \cdots \times X_n$ as  $$ \phi\left(\left(x_1,\ldots,x_{n-1}\right),x_n\right) = \left(x_1,\ldots,x_{n-1}, x_n\right).$$ It is certainly a bijection (since it is nearly the identity function, it is easy to show). The inverse is easy to write down as well: $$ \phi^{-1}\left(\left(x_1,\ldots,x_n\right)\right) = \left(\left(x_1,\ldots,x_{n-1}\right),x_n\right).$$ 
	
	To show continuity, let $U$ be open in $X_1\times \cdots \times X_n$. Then if $x\in \phi^{-1}\left(U\right)$, then $\phi\left(x\right) \in U$ and so there is an open set $U_i$ for each $i$ so that $$ \phi\left(x\right) = \left(x_1,\ldots,x_n\right) \in U_i \times \ldots \times U_n \subset U $$ In particular, we have $$\left(x_1,\ldots,x_{n-1}\right) \in U_1 \times \ldots U_{n-1} $$ and $$ x_n \in U_n,$$ so that $$\phi^{-1}\left(\left(x_1,\ldots,x_n\right)\right) = \left(\left(x_1,\ldots,x_{n-1}\right),x_n\right) \in \left(U_1\times \ldots \times U_{n-1}\right) \times U_n \subset U$$, hence $\phi^{-1}\left(U\right)$ is open and so $\phi$ is continuous. A similar computation shows that $phi^{-1}$ is also continuous, and so $\phi$ is a homeomorphism between the two product spaces.
\end{proof}


\begin{problem}{19.5}
	One of the implications stated in Theorem 19.6 holds for the box topology. Which one?
\end{problem}
\begin{proof}
	The counterexample given in Example 2 demonstrates a function into an infinite Cartesian product space which has continuous components but is not itself continuous, in the box topology. Hence, by process of elimination we know that the previous direction must be the one which holds: a continuous function into the product space will have continuous components.
	
	In fact, let $f:A\rightarrow \prod_{\alpha \in J} X_{\alpha}$ be as in the theorem hypothesis and consider some $\beta \in J$. Let $U_{\beta}\subset X_{\beta}$ be open. Then given $x_{\beta} \in U_{\beta}$, we have a basis element $s$ with  $$ \pi_{\beta}^{-1}\left(U_{\beta}\right) = X_1 \times \cdots \times U_{\beta} \times \cdots $$ This is a subbasis element 
\end{proof}




\begin{problem}{20.1}
	\begin{enumerate}[label=(\alph*)]
		\item In $\mathbb{R}^n$, define $$ d'\left(x,y\right) = \left|x_1-y_1\right| + \cdots + \left|x_n-y_n\right|.$$ Show that $d'$ is a metric that induces the usual topology of $\mathbb{R}^n$. Sketch the basis elements under $d'$ when $n=2$.
		
		\item More generally, given $p\geq 1$, define $$d'\left(x,y\right) = \left[ \sum_{i=1}^n \left|x_i - y_i\right|^p \right]^{1/p} $$ for $x,y\in \mathbb{R}^n$. Assume that $d'$ is a metric. Show that it induces the usual topology on $\mathbb{R}^n$.
	\end{enumerate}
\end{problem}
\begin{proof}
	\begin{enumerate}[label=(\alph*)]
		\item We first must show that $d'$ is a metric. It is clearly non-negative-valued, and assigns zero only when each of the terms in the sum are zero, i.e., when $x=y$. It is also clearly symmetric thanks to the symmetry of the absolute values. It also satisfies the triangle inequality, since \begin{align*}
		d'\left(x,y\right) & = \sum_{i=1}^n \left|x_i - y_i\right| \\
		& = \sum_{i=1}^n \left|\left(x_i - z_i\right) + \left(z_i-y_i\right)\right| \\
		& \leq \sum_{i=1} \left|x_i - z_i\right| + \sum_{i=1}^n \left|z_i -y_i\right| \\ 
		& = d'\left(x,z\right) + d'\left(z,y\right)
		\end{align*}
	\end{enumerate}
	Thus, $d'$ is a metric on $\mathbb{R}^n$.
	
	To show that it induces the same topology on $\mathbb{R}^n$ as the standard Euclidean metric, we need to show that exactly the same sets are open in both topologies. The most straightforward way is to show that the basis elements, i.e., open balls, contain one another (this is the content of Lemma 20.2). {\color{red}In particular, the lemma concerns open balls centered at the same point!} To start, let $B_{d}\left(x,\delta\right)$ be a basis element in the topology induced by the metric $d$. If we take $\epsilon = \delta / \sqrt{n}$, then we can show that $B_{d'}\left(x,\epsilon\right) \subset B_{d}\left(x,\delta\right)$. Let $y \in B_{d'}\left(x,\epsilon\right)$; then $$ d'\left(x,y\right) = \sum_{i=1}^n \left|x_i - y_i \right| < \epsilon/\sqrt{n} $$ so that, in particular, $$ \left|x_i - y_i\right| < \epsilon/\sqrt{n}$$ for every $i$. Then \begin{align*}
	d\left(x,y\right)^2 & = \sum_{i=1}^n \left|x_i - y_i\right|^2\\
	& \leq \sum_{i=1}^n \epsilon^2 / n \\
	& = \epsilon^2
	\end{align*}
	so that $d\left(x,y\right) < \epsilon$, i.e., $y \in B_{d}\left(x,\epsilon\right)$. Thus, the topology induced by $d$ is finer. 
	
	Conversely, consider $B_{d'}\left(x,\delta\right)$. Now, take $\delta = \epsilon/n$. Then if $y \in B_{d}\left(x,\delta\right)$, then $$ d\left(x,y\right) = \left(\sum_{i=1}^n \left|x_i-y_i\right|^2\right)^{1/2} < \delta $$ so that $$ \left|x_i - y_i\right| < \delta = \epsilon/n $$ for all $i$. Then
\begin{align*}		
	d'\left(x,y\right) & = \sum_{i=1}^n \left|x_i-y_i\right| \\
	& < \sum_{i=1}^n \epsilon/n \\
	& = \epsilon
\end{align*}
so that now we have the topology induced by $d'$ the finer. Taken together, the topologies are equal.

	\item The process for this new metric is similar to the previous part of the problem. On the first hand, let $B_d\left(x,\epsilon\right)$ be some open ball in the $d$-induced metric topology on $\mathbb{R}^n$. Then we again have \begin{align*}
	\left|x_j-y_j\right|^p & \leq \sum_{i=1}^n \left|x_i - y_i\right|^p \\
	&  = d'\left(x,y\right)^p
	\end{align*} for any $x,y\in \mathbb{R}^n$ so that $$ \left|x_j - y_j\right| \leq d'\left(x,y\right)$$ which is the same as in the case of $p=1$. Hence, if we choose $\delta = \epsilon/\sqrt{n}$, we can take $y \in B_{d'}\left(x,\delta\right)$ and then we have $$ \left|x_j - y_j\right| \leq d'\left(x,y\right) < \epsilon/\sqrt{n} $$ for every $j=1,\ldots,n$, and so
	\begin{align*}
	d\left(x,y\right)^2 & = \sum_{i=1}^n \left|x_i - y_i\right|^2 \\
	& <\sum_{i=1}^n \epsilon^2 / n \\
	& = \epsilon^2
	\end{align*}
	so that $d\left(x,y\right)<\epsilon$ and we have $y \in B_d\left(x,\epsilon\right)$. We have shown that $B_{d'}\left(x,\delta\right) \subseteq B_d\left(x,\epsilon\right)$, so that the $d'$-induced metric topology is finer than the standard metric topology on $\mathbb{R}^n$.
	
	Conversely, let $B_{d'}\left(x,\epsilon\right)$ be an arbitrary open $d'$-ball. Then we can take $\delta = \epsilon/n^{1/p}$ (as in part (\textit{a})) to obtain a ball $B_{d}\left(x,\delta\right)$ which is inside the former ball. This is true since if $y \in B_{d}\left(x,\delta\right)$, the
	\begin{align*}
		d'\left(x,y\right)^p & = \sum_{i=1}^n \left|x_i-y_i\right|^p \\
		& < n \delta^p \\
		& = \epsilon^p
	\end{align*}
	so that $d'\left(x,y\right) < \epsilon$, i.e., $B_d\left(x,\delta\right) \subseteq B_{d'}\left(x,\epsilon\right)$. Then the $d$-induced topology is finer than the $d'$-induced one. By showing both directions, the topologies are the same.
\end{proof}









\begin{problem}{20.4}
	Let $X$ be a metric space with metric $d$.
	\begin{enumerate}[label=(\alph*)]
		\item Show that $d:X\times X \rightarrow \mathbb{R}$ is continuous.
		\item Let $X'$ denote a space having the same underlying set as $X$. Show that if $d:X'\times X' \rightarrow \mathbb{R}$ is continuous, then the topology of $X'$ is finer than the topology of $X$.
	\end{enumerate}
\end{problem}
\begin{proof}
	\begin{enumerate}[label=(\alph*)]
		\item Let $U\subset \mathbb{R}$ be an open set. Then we must show that $$d^{-1}\left(U\right) = \left\{ \left(x,y\right) \in X\times X \, \middle| \, d\left(x,y\right) \in U \right\}$$ is an open set. Let $\left(x,y\right)\in U$; then $d\left(x,y\right) \in U$. Say $u = d\left(x,y\right)$; then there is some open interval $\left(u-\epsilon, u+\epsilon\right) \subset U$. Then take $x'\times y' \in  B_d\left(x,\epsilon/2\right)\times B_d\left(y,\epsilon/2\right)$. We want to show that $d\left(x',y'\right) \in \left(u-\epsilon,u+\epsilon\right)$. From the triangle equality, $$ d\left(x',y'\right) \leq d\left(x',x\right) + d\left(y,y'\right) < \epsilon $$ so that $$ \left|d\left(x',y'\right)-u\right| < \epsilon $$ and so $d\left(x',y'\right)\in U$. We have shown that  $B_d\left(x,\epsilon/2\right)\times B_d\left(y,\epsilon/2\right) \subseteq d^{-1}\left(U\right)$, and this set is clearly open as the product of two basis elements. Thus, $d$ is continuous.
		\item Let $U$ be open in the topology on $X$. 
	\end{enumerate}
\end{proof}









% --------------------------------------------------------------
%     You don't have to mess with anything below this line.
% --------------------------------------------------------------

\end{document}
