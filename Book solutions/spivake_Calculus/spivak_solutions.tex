% --------------------------------------------------------------
% This is all preamble stuff that you don't have to worry about.
% Head down to where it says "Start here"
% --------------------------------------------------------------

\documentclass[12pt]{article}

\usepackage[margin=1in]{geometry}
\usepackage{amsmath,amsthm,amssymb,scrextend}
\usepackage{fancyhdr}
\pagestyle{fancy}
\usepackage{xcolor}
\usepackage{enumitem} % to modify enum labels

\newcommand{\N}{\mathbb{N}}
\newcommand{\Z}{\mathbb{Z}}
\newcommand{\I}{\mathbb{I}}
\newcommand{\R}{\mathbb{R}}
\newcommand{\Q}{\mathbb{Q}}
\renewcommand{\qed}{\hfill$\blacksquare$}
\let\newproof\proof
\renewenvironment{proof}{\begin{addmargin}[1em]{0em}\begin{newproof}}{\end{newproof}\end{addmargin}\qed}
% \newcommand{\expl}[1]{\text{\hfill[#1]}$}

\newenvironment{theorem}[2][Theorem]{\begin{trivlist}
\item[\hskip \labelsep {\bfseries #1}\hskip \labelsep {\bfseries #2.}]}{\end{trivlist}}
\newenvironment{lemma}[2][Lemma]{\begin{trivlist}
\item[\hskip \labelsep {\bfseries #1}\hskip \labelsep {\bfseries #2.}]}{\end{trivlist}}
\newenvironment{problem}[2][Problem]{\begin{trivlist}
\item[\hskip \labelsep {\bfseries #1}\hskip \labelsep {\bfseries #2.}]}{\end{trivlist}}
\newenvironment{exercise}[2][Exercise]{\begin{trivlist}
\item[\hskip \labelsep {\bfseries #1}\hskip \labelsep {\bfseries #2.}]}{\end{trivlist}}
\newenvironment{reflection}[2][Reflection]{\begin{trivlist}
\item[\hskip \labelsep {\bfseries #1}\hskip \labelsep {\bfseries #2.}]}{\end{trivlist}}
\newenvironment{proposition}[2][Proposition]{\begin{trivlist}
\item[\hskip \labelsep {\bfseries #1}\hskip \labelsep {\bfseries #2.}]}{\end{trivlist}}
\newenvironment{corollary}[2][Corollary]{\begin{trivlist}
\item[\hskip \labelsep {\bfseries #1}\hskip \labelsep {\bfseries #2.}]}{\end{trivlist}}



\begin{document}

% --------------------------------------------------------------
%                         Start here
% --------------------------------------------------------------

\lhead{Spivak ``Calculus'' Solutions}
\chead{Stan Tuznik}
\rhead{\today}

% \maketitle


\section*{Chapter 1 Basic Properties of Numbers}

\begin{problem}{1.1}
Prove the following;
\begin{enumerate}[label=(\roman*)]
\item if $ax=a$ for some number $a\neq 0$, then $x=1$.
\item $x^2-y^2 = \left(x-y\right)\left(x+y\right)$.
\item If $x^2 =y^2$, then $x=y$ or $x=-y$.
\item $x^3 - y^3 = \left(x-y\right)\left(x^2+xy+y^2\right)$.
\item $x^n - y^n = \left(x-y\right)\left(x^{n-1} + x^{n-2}y + \cdots + xy^{n-2} + y^{n-1}\right)$.
\item $x^3 + y^3 = \left(x+y\right)\left(x^2-xy+y^2\right)$. (There is a particularly easy way to do this, using $\left(iv\right)$, and it will show you how to find a factorization for $x^n+y^n$ whenever $n$ is odd.)
\end{enumerate}
\end{problem}

\begin{proof}
\begin{enumerate}[label=(\roman*)]
	\item Assume $ax=a$ for some $a\neq 0$. Then $a^{-1}$ exists, and we have
	\[ \begin{split} x & = 1\cdot x \\ & = \left(a^{-1}a\right)x \\ &= a^{-1} \left(ax\right) \\ &= a^{-1} a \\ & = 1\end{split}\]
	\item \[ \begin{split} \left(x-y\right)\left(x+y\right) & = \left(x-y\right)x + \left(x-y\right)y \\ & = x^2 - yx - +xy - y^2 \\ & = x^2 - y^2 \end{split} \]
	\item $\ldots$ 
	\item \[ \begin{split} \left(x-y\right)\left(x^2 + xy + y^2 \right) &=  x\left(x^2 + xy + y^2\right) - y \left(x^2 + xy + y^2\right) \\ & = x^3 + x^2 y + x y^2 - x^2y - xy^2 - y^3 \\ &= x^3 - y^3\end{split} \]
	\item We can prove this just by manipulating the products, but it is more instructive to prove it by induction. The base case was demonstrated in part $\left(iv\right)$ of the problem. For the inductive step, assume it holds for some $n$, i.e., that \[ x^n - y^n = \left(x-y\right)\left(x^{n-1} + x^{n-2}y + \cdots + xy^{n-2} + y^{n-1}\right) \] Then
	\begin{equation*}\begin{split}
		 \left(x^{n} + x^{n-1}y + \cdots + xy^{n-1} + y^{n}\right) &= x\left(x^{n-1} + x^{n-2}y + \cdots + y^{n-1} + x^{-1}y^{n}\right) \\
		 & = x \left(x^{n-1} + x^{n-2}y + \cdots + y^{n-1} \right) + y^n \\
	\end{split}\end{equation*} Thus,  multiplying both sides by $\left(x-y\right)$, we have
	\begin{equation*}\begin{split}
		x \left(x-y\right) \left(x^{n-1} + x^{n-2}y + \cdots + y^{n-1} \right) + \left(x-y\right) y^n & = x \left(x^n-y^n\right) + xy^n - y^{n+1} \\
		& = x^{n+1} - xy^n + xy^n - y^{n+1} \\ 
		& = x^{n+1} - y^{n+1}
	\end{split}\end{equation*} which is what we sought to prove.
	\item This can be easily demonstrated by simply multiplying everything out, but as the hint suggests, we can do it more interestingly by comparing with $\left(iv\right)$. In particular, let $y=-z$ and consider $x^3 - z^3$. Then from $\left(iv\right)$ we have
	\begin{equation*}\begin{split}
		x^3 + y^3 & = x^3 - z^3 \\
		& = \left(x-z\right) \left(x^2 + xz + z^2\right) \\
		& = \left(x+y\right) \left(x^2 - xy + y^2 \right)
	\end{split} \end{equation*} The suggested factorization for $x^n + y^n$ for $n$ odd involves $\ldots$
\end{enumerate}


\end{proof}



\begin{problem}{1.2}
What is wrong with the following ``proof''? Let $x=y$. Then
\begin{equation*}\begin{split}
x^2 & = xy, \\
x^2 - y^2 & =xy-y^2, \\
\left(x+y\right)\left(x-y\right) & =y\left(x-y\right),\\
x+y & = y, \\
2y & =y, \\
2 &= 1.
\end{split}\end{equation*}
\end{problem}
\begin{proof}
The ``proof'' moves from the third line to the fourth line by ``cancelling out'' the common factor of $\left(x-y\right)$ from both sides. In actuality, this is accomplished by multiplying both sides by the multiplicative inverse of $\left(x-y\right)$, $\left(x-y\right)^{-1}$. However multiplicative inverses only exist for non-zero numbers, and $\left(x-y\right)=0$ since $x=y$. Thus, this inverse does not exist and we cannot perform this cancellation.
\end{proof}






\begin{problem}{1.3}
Prove the following:
\begin{enumerate}[label=(\roman*)]
	\item $\frac{a}{b} = \frac{ac}{bc},$ if $b,c \neq 0$.
	\item $\frac{a}{b} + \frac{c}{d} = \frac{ad+bc}{bd},$ if $b,d\neq 0$.
	\item $\left(ab\right)^{-1} = a^{-1}b^{-1},$ if $a,b\neq 0$. (To do this you must remember the defining property of $\left(ab\right)^{-1}$.)
	\item $\frac{a}{b}\cdot \frac{c}{d} = \frac{ac}{db}$, if $b,d\neq 0$.
	\item $\frac{a}{b} / \frac{c}{d} = \frac{ad}{bc},$ if $b,c,d\neq 0$.
	\item If $b,d\neq 0$, then $\frac{a}{b} = \frac{c}{d}$ if and only if $ad=bc$. Also determine when $\frac{a}{b}=\frac{b}{a}$.
\end{enumerate}
\end{problem}
\begin{proof}
\begin{enumerate}[label=(\roman*)]
	\item
	\item
	\item Recall that for nonzero $a,b$, the inverse $\left(ab\right)^{-1}$ is defined by the equation \[ \left(ab\right)\left(ab\right)^{-1} = 1 \] and that inverses are unique. Then notice that \[ \begin{split} \left(ab\right) \left(a^{-1}b^{-1}\right) & = \left(aa^{-1}\right)\left(bb^{-1}\right) \\ & = 1\cdot 1 \\ & = 1 \end{split} \] where we used both the commutativity and associativity of multiplication. Thus, $a^{-1}b^{-1} = \left(ab\right)^{-1}$ is the unique inverse of $ab$.
	\item
	\item
	\item
\end{enumerate}
\end{proof}


\begin{problem}{1.4}
Find all numbers $x$ for which
\begin{enumerate}[label=(\roman*)]
	\item $4-x < 3-2x$.
	\item $5-x^2 < 8$.
	\item $5-x^2 < -2$.
	\item $\left(x-1\right)\left(x-3\right) > 0$. (When is a product of two numbers positive?)
	\item $x^2 - 2x+2 > 0$.
	\item $x^2+x+1>2$.
	\item $x^2 -x+10 > 16$.
	\item $x^2+x+1>0$.
	\item $\left(x-\pi\right)\left(x+5\right)\left(x-3\right) > 0$.
	\item $\left(x-\sqrt[3]{2}\right)\left(x-\sqrt{2}\right) > 0$.
	\item $2^x < 8$.
	\item $x+3^x < 4$.
	\item $\frac{1}{x} + \frac{1}{1-x} > 0$.
	\item $\frac{x-1}{x+1} > 0$.
\end{enumerate}
\end{problem}

\begin{problem}{1.5}
Prove the following:
\begin{enumerate}[label=(\roman*)]
	\item If $a<b$ and $c<d$, then $a+c < b+d$.
	\item If $a<b$, then $-b < -a$.
	\item If $a<b$ and $c>d$, then $a-c<b-d$.
	\item If $a<b$ and $c>0$, then $ac < bc$.
	\item If $a<b$ and $c<0$, then $ac >bc$.
	\item If $a>1$, then $a^2 > a$.
	\item If $0<a<1$, then $a^2<a$.
	\item If $0\leq a < b$ and $0\leq c < d$, then $ac<bd$.
	\item If $0\leq a < b$, then $a^2 < b^2$. (Use $\left(viii\right)$.)
	\item If $a,b\geq 0$ and $a^2 < b^2$, then $a<b$. (Use $\left(ix\right)$, backwards.)
\end{enumerate}
\end{problem}



\begin{problem}{1.6}
\begin{enumerate}[label=(\roman*)]
	\item Prove that if $0\leq x < y$, then $x^n < y^n$, $n=1,2,3,\ldots$.
	\item Prove that if $x<y$ and $n$ is odd, then $x^n<y^n$.
	\item Prove that if $x^n = y^n$ and $n$ is odd, then $x=y$.
	\item Prove that if $x^n = y^n$ and $n$ is even, then $x=y$ or $x=-y$.
\end{enumerate}
\end{problem}
\begin{proof}
\begin{enumerate}[label=(\roman*)]
	\item Assume $0\leq x < y$. Then certainly $x^1 < y^1 $. Assume that $x^n < y^n$ for some integer $n>1$. Then \[ \begin{split} x^{n+1} & = x\cdot x^n \\ & < x \cdot y^n \\ & < y \cdot y^n \\ & = y^{n+1} \end{split} \] where we used both the inductive hypothesis and the base case to assert the inequalities.
\end{enumerate}
\end{proof}



\begin{problem}{1.7}
Prove that if $0<a<b$, then $$ a < \sqrt{ab} < \frac{a+b}{2} < b.$$ Notice that the inequality $\sqrt{ab} \leq \left(a+b\right)/2$ holds for all $a,b\geq 0$. A generalization of this fact occurs in Problem 2-22.
\end{problem}

\begin{problem}{1.8}
Although the basic properties of inequalities were stated in terms of the collection $P$ of all positive numbers, and $<$ was defined in terms of $P$, this procedure can be reversed. Suppose that $P10-P12$ are replaced by
\begin{enumerate}[label=($P'$\arabic*),start=10]
	\item For any numbers $a$ and $b$ one, and only one, of the following holds:
	\begin{enumerate}[label=(\roman*)]
		\item $a=b$,
		\item $a<b$,
		\item $b<a$.
	\end{enumerate}
	\item For any numbers $a$, $b$, and $c$, if $a<b$ and $b<c$, then $a<c$.
	\item For any numbers $a$, $b$, and $c$, if $a<b$, then $a+c < b+c$.
	\item For any numbers $a$, $b$, and $c$, if $a<b$ and $0<c$, then $ac<bc$.
\end{enumerate}
Show that $P10-P12$ can be deduced as theorems.
\end{problem}







% --------------------------------------------------------------
%     You don't have to mess with anything below this line.
% --------------------------------------------------------------

\end{document}
