% --------------------------------------------------------------
% This is all preamble stuff that you don't have to worry about.
% Head down to where it says "Start here"
% --------------------------------------------------------------

\documentclass[12pt]{article}

\usepackage[margin=1in]{geometry}
\usepackage{amsmath,amsthm,amssymb,scrextend}
\usepackage{fancyhdr}
\pagestyle{fancy}
\usepackage{xcolor}
\usepackage{enumitem}

\newcommand{\N}{\mathbb{N}}
\newcommand{\Z}{\mathbb{Z}}
\newcommand{\I}{\mathbb{I}}
\newcommand{\R}{\mathbb{R}}
\newcommand{\Q}{\mathbb{Q}}
\renewcommand{\qed}{\hfill$\blacksquare$}
\let\newproof\proof
\renewenvironment{proof}{\begin{addmargin}[1em]{0em}\begin{newproof}}{\end{newproof}\end{addmargin}\qed}
% \newcommand{\expl}[1]{\text{\hfill[#1]}$}

\newenvironment{theorem}[2][Theorem]{\begin{trivlist}
\item[\hskip \labelsep {\bfseries #1}\hskip \labelsep {\bfseries #2.}]}{\end{trivlist}}
\newenvironment{lemma}[2][Lemma]{\begin{trivlist}
\item[\hskip \labelsep {\bfseries #1}\hskip \labelsep {\bfseries #2.}]}{\end{trivlist}}
\newenvironment{problem}[2][Exercise]{\begin{trivlist}
\item[\hskip \labelsep {\bfseries #1}\hskip \labelsep {\bfseries #2.}]}{\end{trivlist}}
\newenvironment{exercise}[2][Exercise]{\begin{trivlist}
\item[\hskip \labelsep {\bfseries #1}\hskip \labelsep {\bfseries #2.}]}{\end{trivlist}}
\newenvironment{reflection}[2][Reflection]{\begin{trivlist}
\item[\hskip \labelsep {\bfseries #1}\hskip \labelsep {\bfseries #2.}]}{\end{trivlist}}
\newenvironment{proposition}[2][Proposition]{\begin{trivlist}
\item[\hskip \labelsep {\bfseries #1}\hskip \labelsep {\bfseries #2.}]}{\end{trivlist}}
\newenvironment{corollary}[2][Corollary]{\begin{trivlist}
\item[\hskip \labelsep {\bfseries #1}\hskip \labelsep {\bfseries #2.}]}{\end{trivlist}}



\begin{document}

% --------------------------------------------------------------
%                         Start here
% --------------------------------------------------------------

\lhead{Dummit \& Foote solutions}
\chead{Stan Tuznik}
\rhead{\today}

% \maketitle


\section*{Chapter 2 Basic Topology}


\begin{problem}{2.6}
Let $E'$ be the set of all limit points of a set $E$. Prove that $E'$ is closed. Prove that $E$ and $\overline{E}$ have the same limit points. (Recall that $\overline{E}=E\cup E'$.) Do $E$ and $E'$ always have the same limit points?
\end{problem}
\begin{proof}
Let $x$ be a limit point of $E'$. We want to show that $x \in E'$. Since $x$ is a limit point, for any neighborhood $N$ of $x$, we have some point $y \in N\cap E'$. That is, $y$ is a limit point of $E$. Then since $N$ is a neighborhood of $y$, there is some point $z \in N\cap E$. But then for any arbitrary neighborhood of $x$, $N$, we have found a point $z \in E$ with $z \in N$. Thus, $x$ is a limit point of $E$, i.e., $x\in E'$. That is, $E'$ contains all of its limit points, and is closed by definition.

{\color{red} We can alternatively approach this problem from the equivalent definition of a closed set as a complement of an open set. Let $x\notin E'$. Then $x$ is not a limit point of $E$. That is, there is some neighborhood $N$ of $x$ which does not intersect $E$: $N\cap E = \varnothing$. We claim also that $N\cap E' = \varnothing$. Otherwise, if $z \in N \cap E'$, then since $z \in E'$, and $N$ is a neighborhood of $z$, there is some point $p \in E$ with $p \in N$. But now we have a point $p$ of $E$ in our neighborhood $N$ of $x$, which we constructed to have no such points of $E$. $\bot$ Hence we have $N\cap E' = \varnothing$, and so $N$ is a neighborhood of $x$ which does not intersect $E'$. Thus, $E'$ is closed since $E'^c$ is open.}

Let $x$ be a limit point of $E$. Then let $N$ be a neighborhood of $x$. Then there is a $y\in E$ with $y\in N$. But then $y \in E \subset E\cup E' = \overline{E}$, so arbitrary neighborhoods of $x$ contain points of $\overline{E}$. Thus, $x$ is a limit point of $\overline{E}$. Conversely, let $x$ be a limit point of $\overline{E}$, with $N$ a neighborhood of $x$. Then $x$ contains some point of $\overline{E}$, say $y$. If $y\in E$, then $N$ trivially contains a point $y$ of $E$. Otherwise, if $y \in E'$, then since $N$ is a neighborhood of $y$, $N$ contains some point $z \in E$. Thus, in either case, neighborhood $N$ of $x$ contains a point of $E$; $x$ is a limit point of $E$. From both directions, we see that the limit points of $E$ and of $\overline{E}$ are the same.

Let $E = \left\{ \frac{1}{n} \, | \, n\in \mathbb{N} \right\}$. Then surely $E' = \left\{0\right\}$, which has no limit points. But since $0$ is a limit point of $E$, $E$ and $E'$ do not have the same limit points.
\end{proof}



\begin{problem}{2.7}
Let $A_1$, $A_2$, $A_3$, \ldots be subsets of a metric space.
\begin{enumerate}[label=(\alph*)]
	\item If $B_n = \cup_{i=1}^n A_i$, prove that $\overline{B}_n = \cup_{i=1}^n \overline{A}_i$ for $n=1,2,3,\ldots$.
	\item If $B = \cup_{i=1}^{\infty} A_i$, prove that $\overline{B} \supset \cup_{i=1}^{\infty} \overline{A}_i$.
\end{enumerate}
Show, by an example, that this inclusion can be proper.
\end{problem}

\begin{proof}
\begin{enumerate}[label=(\alph*)]
	\item Let $x \in \overline{B}_n$. 
\end{enumerate}
\end{proof}




\begin{problem}{2.10}
Let $X$ be an infinite set. For $p\in X$ and $q\in X$, define
\begin{equation*}
 d\left(p,q\right) = \left\{ \begin{array}{lr} 1 & \left(\text{if}\, p\neq q\right) \\ 0 & \left(\text{if}\, p=q\right) \end{array} \right.
\end{equation*}
Prove that this is a metric. Which subsets of the resulting metric space are open? Which are closed? Which are compact?
\end{problem}

\begin{proof}
In order for this to be a metric, it must be non-negative and $d\left(p,q\right) = 0$ only when $p=q$. This is obviously true from the definition of $d$. It is also very clearly symmetric. The final condition is that the triangle inequality must hold for every combination of points $x,y,z \in X$. First, if $x=z$, then $d\left(x,z\right) =0 \leq d\left(x,y\right) + d\left(y,z\right)$ since the right hand side is at least $0$. On the other hand, if $x\neq z$, then $d\left(x,z\right)=1$. If $y\in X$ is such that $y\neq x$ and $y\neq z$, then $d\left(x,y\right) + d\left(y,z\right) = 2$, so the inequality holds. Instead, if $y\neq x$ but $y=z$, then $d\left(x,y\right) + d\left(y,z\right) = 1$ and the inequality holds; this same logic applies if $y\neq x$ but $y=z$. We cannot have both $y=x$ and $y=z$, since we assumed $x\neq z$. Thus in all cases we have the triangle inequality satisfied. This completes the proof that $d$ is a metric. {\color{red}It is the simplest metric, the discrete metric on $X$.}

A set $A\subset X$ is open {\color{red}in the metric space sense} if every point $x\in A$ has some open ball neighborhood $B_r\left(x\right) = \left\{ y\in X\,|\, d\left(x,y\right)<r\right\}$ with $B_r\left(x\right) \subset A$. Note that for any $x\in X$ we have the ball neighborhood $\left\{x\right\} = B_1\left(x\right)$ which is certainly a subset of and set $A$ containing $x$. Thus every set is open in this metric space. Since any closed set is the complement of some open set, every set is also closed in this metric space.

For set $K \subset X$, we have the open cover $\left\{ \left\{k\right\}\, | \, k \in K\right\}$ since the singleton sets are open. This open cover clearly has {\color{red}no subcover}, since removing any one of the singletons would remove the corresponding point from $K$. Thus, the open cover of $K$ has a finite open subcover if and only if this cover itself is the finite open subcover, i.e., if $K$ is a finite set. Thus, the finite subsets of $K$ are the only compact sets in this metric.
\end{proof}






\begin{problem}{2.12}
Let $K\subset \mathbb{R}^1$ consist of $0$ and the numbers $1/n$, for $n=1,2,3,\ldots $. Prove that $K$ is compact directly from the definition (without using the Heine-Borel theorem).
\end{problem}
\begin{proof}
{\color{red}The Heine-Borel theorem applies here since $K$ is certainly closed and bounded. Since we cannot use this, we must take an arbitrary open cover of $K$ and construct a finite open subcover. By drawing some pictures, we note that $0$ is a limit point of $K$, and so any open neighborhood $N$ about $0$ must naturally contain some point $1/m$ to the right of it. Then we can use this open set $N$ to cover the infinitely many points ``to the left of'' $1/m$. } Let $\left\{U_{\alpha}\right\}_{\alpha =1}^{\infty}$ be an open cover of $K$. Then $0\in U_{\beta}=:V $ for one of these open sets, which we will call $V$. Then since $0$ is a limit point of $K$ {\color{red}(easy to show)}, we have some point $1/m \in K$ also in $V$. Since we are considering the metric space $\mathbb{R}^1$, we have $1/m' \in V$ for all $ m' $ such that $m' \geq m$. For each $1/n \in K$ with $1\leq n < m$, we have some element of the open cover, say $U_{\gamma}=:V_n$ with $1/n \in V_n$. Then $\cup_{i=1}^{m-1} V_n \cup V $ is a finite open subcover of $K$. Thus, $K$ is compact.
\end{proof}



\begin{problem}{2.13}
Construct a compact set of real numbers whose limit points form a countable set.
\end{problem}




\begin{problem}{2.14}
Give an example of an open cover of the segment $\left(0,1\right)$ which has no finite subcover.
\end{problem}


\begin{problem}{2.16}
Regard $\mathbb{Q}$, the set of all rational numbers, as a metric space, with $d\left(p,q\right)=\left|p-q\right|$. Let $E$ be the set of all $p\in \mathbb{Q}$ such that $2 < p^2 < 3$. Show that $E$ is closed and bounded in $\mathbb{Q}$, but that $E$ is not compact. Is $E$ open in $\mathbb{Q}$?
\end{problem}
\begin{proof}
Let $\mathbb{Q}$ be the metric space described in the problem statement, and let $E = \left\{p\in \mathbb{Q}\,|\, 2<p^2<3\right\}$. Certainly $2$ and $-2$ are upper and lower bounds for $E$, respectively, since $2^2 =4 > 3$; hence, $E$ is bounded. 

Let $x \notin E$. Then $x\in \mathbb{Q}$ but either $x^2 \leq 2$ or $x^3 \geq 3$. Consider first the case in which $x^2 \leq 2$. Certainly, then, $x^2<2 $ since $x^2=2$ for no rational $x$. 

\end{proof}






\begin{problem}{2.17}
Let $E$ be the set of all $x\in\left[0,1\right]$ whose decimal expansion contains only the digits $4$ and $7$. Is $E$ countable? Is $E$ dense in $\left[0,1\right]$? is $E$ compact? Is $E$ perfect?
\end{problem}


\begin{problem}{2.18}
Is there a nonempty perfect set in $\mathbb{R}^1$ which contains no rational number?
\end{problem}
\begin{proof}
{\color{red}Recall that a perfect set is one which is closed every point is a limit point.} 
\end{proof}




\begin{problem}{2.19}
\begin{enumerate}[label=(\alph*)]
	\item If $A$ and $B$ are disjoint closed sets in some metric space $X$, prove that they are separated.
	\item Prove the same for disjoint open sets.
	\item Fix $p\in X$, $\delta > 0$, define $A$ to be the set of all $q\in X$ for which $d\left(p,q\right)<\delta$, define $B$ similarly, with $>$ in place of $<$. Prove that $A$ and $B$ are separated.
	\item Prove that every connected metric space with at least two points is uncountable. \textit{Hint}: Use (c).
\end{enumerate}
\end{problem}

\begin{proof}
{\color{red}Recall that two sets $A$ and $B$ are separated if both $A\cap \overline{B}$ and $\overline{A}\cap B$ are empty.}
\begin{enumerate}[label=(\alph*)]
	\item Let $A$ and $B$ be disjoint and closed. By definition of closed, $A$ contains all of its limit points and $B$ contains all of its limit points. Thus, $A=\overline{A}$ and $B=\overline{B}$. Since $A$ and $B$ are disjoint, we have $A\cap \overline{B} = A\cap B = \varnothing$ and $\overline{A}\cap B = A\cap B = \varnothing$. That is, disjoint closed sets in a metric space are separated.
	\item {\color{red}For disjoint open sets, we cannot use the same trick with the limit points, since open sets do not necessarily contain their limit points.} Assume by way of contradiction that $B$ contains some limit point of $A$, say $x$. Then since $B$ open, there is some open neighborhood $U_x$ of $x$ with $U_x \subset B$. But since $x$ is a limit point of $A$, the neighborhood $U_x$ contains some point $y$ of $A$. But then we have $y \in U_x \subset B$, and $A$ and $B$ are assumed to be disjoint $\bot$. Thus, neither $A$ nor $B$ may contain a limit point of the other, and so $A$ and $B$ are separated.
	\item Let $A= \left\{ q\in X\, | \, d\left(p,q\right) < \delta \right\}${\color{red}; this is an open ball about $p$ in the $d$ metric, and it is certainly open (in that metric)}. Similarly, $B$ is the open complement of $A$, and is easily also open. Then $A$ and $B$ are disjoint open sets in metric space $X$, and so by (b) the two are separated.
	\item Let $X$ be a connected metric space containing two distinct points $x$ and $y$. Let $\delta = \frac{1}{2}d\left(x,y\right)$ and define $A$ and $B$ as in (c) but with this $\delta >0$. Then $A$ and $B$ separate $X$ unless there is some $z$ with $d\left(x,z\right) = \delta$. Since $X$ connected, this $z$ must exist. We picked $\delta = d\left(x,y\right)/2$ arbitrarily; we might just as easily have chosen $0<\delta <d\left(x,y\right)$ and for each choice of $\delta$ we obtain some $z\in X$. Thus, we have a bijection from $\left(0,1\right)$ to a subset of $X$ (the $z$), and so this subset is uncountable. Thus, $X$ itself must be uncountable.
\end{enumerate}
\end{proof}







\newpage
\section*{Chapter 3 Numerical Sequences and Series}

\begin{problem}{3.1}
Prove that convergence of $\left\{s_n\right\}$ implies convergence of $\left\{\left|s_n\right|\right\}$. is the converse true?
\end{problem}
\begin{proof}
This is true from the triangle inequality. If $\left\{s_n\right\}$ converges, let $\epsilon>0$ and we have some $N\in \mathbb{N}$ so that $n>N \implies \left| s_n - L\right|<\epsilon$, where $L$ is the limit of the sequence. Then for $n>N$, by the triangle inequality we have $\left|\left|s_n\right|-\left|L\right|\right| \leq \left|s_n-L\right| < \epsilon$ and so $\lim_{n\rightarrow \infty} \left|s_n\right| = \left|L\right|$.

The converse is not true. The sequence $s_n = \left(-1\right)^n$ does not converge, but the sequence $\left|s_n\right|=1$ trivially converges.
\end{proof}


\begin{problem}{3.2}
Calculate $\lim_{n\rightarrow \infty} \left(\sqrt{n^2+n}-n\right)$.
\end{problem}
\begin{proof}
We can guess what this limit might be by considering the function $f\left(x\right) = \sqrt{n^2+n}-n$. We can guess the limit as $x\rightarrow \infty$ by the usual trick of rationalizing the denominator:
\begin{equation*}
\begin{split}
\sqrt{x^2+x}-x & = \left( \sqrt{x^2+x}-x\right)\left( \frac{\sqrt{x^2+x}+x}{\sqrt{x^2+x}+x}\right) \\
& = \frac{x}{\sqrt{x^2+x}+x} \\
& = \frac{1}{1+\sqrt{1+1/x}}
\end{split}
\end{equation*}
where the last equality is allowed since $x$ is away from $0$. Without formally taking limits, we can see that as $x$ gets very large, the $1/x$ term tends toward $0$ and the entire quantity then tends toward $1/2$.
\end{proof}






% --------------------------------------------------------------
%     You don't have to mess with anything below this line.
% --------------------------------------------------------------

\end{document}
